\chapter{Processamento de Strings}

\section{Aho-Corasick}

Após inserir todas as strings, chamar a função $aho();$

\begin{multicols}{2}
	\begin{lstlisting}

#define MAXN 100100
#define ALPHA 15

int trie[MAXN][ALPHA];
int term[MAXN];
int failure[MAXN];
int cnt;
void insert(string s){
	int node = 0;
	for(auto c : s){
		if(!trie[node][c-'a'])
		trie[node][c-'a'] = cnt++;
		node = trie[node][c-'a'];
	}
	term[node] = 1;
}

void aho(){
	queue<int> q;
	
	for(int i = 0; i < ALPHA; i++){
		if(trie[0][i]){
			failure[trie[0][i]] = 0;
			q.push(trie[0][i]);
		}
	}
	
	while(!q.empty()){
		int u = q.front(); q.pop();
		for(int i = 0; i < ALPHA; i++){
			if(trie[u][i]){
				int v = failure[u];
				while(v && !trie[v][i])
				v = failure[v];
				v = trie[v][i];
				failure[trie[u][i]] = v;
				term[trie[u][i]] |= term[v];
				q.push(trie[u][i]);
			}	
		}
	}
}

int next(int node, int c){
	while(node && !trie[node][c])
	node = failure[node];
	return trie[node][c];
}

void init(){
	memset(trie, 0, sizeof trie);
	memset(term, 0, sizeof term);
	memset(failure, 0, sizeof failure);
	memset(vis, 0, sizeof vis);
	cnt = 1;
}


\end{lstlisting}
\end{multicols}

\section{Rabin-Karp}

String matching $O(|S| + |T|)$.

\begin{multicols}{2}
	\begin{lstlisting}


string s, t; // input
const int p = 31;
vector<ull> p_pow(max(s.size(), t.size()));
p_pow[0] = 1;
for(int i = 1; i < p_pow.size(); i++)
	p_pow[i] = p_pow[i-1]*p;
	
vector<ull> h(t.size());
for(int i = 0; i < t.size(); i++){
	h[i] = (t[i]-'a'+1) * p_pow[i];
	if(i)
		h[i] += h[i-1];
}

ull h_s = 0;
for(int i = 0; i < s.size(); i++)
	h_s += (s[i]-'a'+1) * p_pow[i];

for(int i = 0; i + s.size()-1 < t.size(); i++){
	ull cur_h = h[i+s.size()-1];
	if(i)
		cur_h -= h[i-1];	
	if(cur_h == h_s * p_pow[i])
		cout << i << " ";
}
\end{lstlisting}
\end{multicols}
\section{Repetend: menor período de uma string}

Menor período de uma string em $O(n)$.

\begin{multicols}{2}
	\begin{lstlisting}

#define MAXN 100010

int repetend(string s){
	int n = s.size();
	int nxt[n+1];
	nxt[0] = -1;
	for(int i = 1; i <= n; i++){
		int j = nxt[i-1];
		while(j>=0 && s[j] != s[i-1])
			j = nxt[j];
		nxt[i] = j+1;
	}
	int a = n-nxt[n];
	if(n%a==0)
		return a;
	return n;	
}

\end{lstlisting}
\end{multicols}
\section{Suffix Array}

\begin{multicols}{2}
\begin{lstlisting}
#define MAXN 100100
int suffix[MAXN];
int RA[MAXN], aux[MAXN], lcp[MAXN], invSuffix[MAXN], c[MAXN];
// Lembrar de colocar o ultimo caractere alguem menor que o alfabeto
// A..Z pode usar $
// a..z pode usar {
string p; 

// O(n^2 logn)
bool cmpSlow(int a, int b){
	return p.substr(a) < p.substr(b);
}
	
void saSlow(){
	for(int i = 0; i < p.size(); i++)
		suffix[i] = i;
	sort(suffix, suffix+p.size(), cmpSlow);
}
	
void lcpSlow(){
	lcp[p.size()-1] = 0;
	for(int i = 0; i < p.size()-1; i++){
		int l = 0;
		while(p[suffix[i]+l] == p[suffix[i+1]+l]) l++;
		lcp[i] = l;
	}
}
	
// O(n log^2(n)) - Se usar o sort()
// O(n logn)     - Se usar countingSort
int n;
void countingSort(int k){
	memset(c, 0, sizeof c);
	int maxi = max(300, n);
	for(int i = 0; i < n; i++)
		c[RA[i+k]]++;
	int sum = 0;
	for(int i = 0; i < maxi; i++){
		int aux = c[i];
		c[i] = sum;
		sum += aux;
	}
	for(int i = 0; i < n; i++)
		aux[c[RA[suffix[i]+k]]++] = suffix[i];
	for(int i = 0; i < n; i++)
		suffix[i] = aux[i];
}
	
void saFast(){
	n = p.size();
	memset(RA, 0, sizeof RA);
	for(int i = 0; i < n; i++){
		suffix[i] = i;
		RA[i] = p[i];
	}
	
	for(int k = 1; k < n; k <<= 1){
		countingSort(k); // Ordena por second
		countingSort(0); // Ordena por first
		
		aux[suffix[0]] = 0;
		int newRa = 0;
		for(int i = 1; i < n; i++){
			if(RA[suffix[i]] == RA[suffix[i-1]] && 	RA[suffix[i]+k] == RA[suffix[i-1]+k])
				aux[suffix[i]] = newRa;
			else
				aux[suffix[i]] = ++newRa;
		}
		for(int i = 0; i < n; i++)
			RA[i] = aux[i];
		if(RA[suffix[n-1]] == n-1)
			break;
	}
}
	
void lcpFast(){
	lcp[n-1] = 0;
	for(int i = 0; i < n; i++)
		invSuffix[suffix[i]] = i;
	
	int k = 0;
	for(int i = 0; i < n; i++){
		if(invSuffix[i] == n-1){
			k = 0;
			continue;
		}
		int j = suffix[invSuffix[i]+1];
		while(i+k<n && j+k<n && p[i+k]==p[j+k]) k++;
		lcp[invSuffix[i]] = k;
		if(k)
			k--;
	}
}
	
// Para LCS de muitas strings
int sz[MAXN];
int owner[MAXN], separators[MAXN];
int words;

void buildOwner(){
	memset(owner, 0, sizeof owner);
	for(int i = 0; i < n; i++){
		if(separators[suffix[i]]){
			owner[i] = -1;
			continue;
		}
		for(int j = 0; j < words; j++)
		if(suffix[i] < sz[j]){
			owner[i] = j;
			break;
		}
	}
}
	
\end{lstlisting}
\end{multicols}

