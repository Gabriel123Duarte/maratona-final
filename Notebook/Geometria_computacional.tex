\chapter{Geometria Computacional}

\section{Template - Júnior}
\begin{multicols}{2}
	\begin{lstlisting}
#define pi acos(-1.0)
#define eps 1e-6

struct Point {
	double x, y;
	Point(){};
	Point(double _x, double _y) {
		x = _x;
		y = _y;
	}
	void read(){ scanf("%lf %lf", &x, &y); }
		double distance(Point other) { return hypot(x - other.x, y - other.y); }
		Point operator + (Point other) { return Point(x + other.x, y + other.y); }
		Point operator - (Point other) { return Point(x - other.x, y - other.y); }
		Point operator * (double t) { return Point(x * t, y * t); }
		Point operator / (double t) { return Point(x / t, y / t); }
		double operator * (Point q) {return x * q.x + y * q.y;} //a*b = |a||b|cos(ang) //Positivo se o vetor B esta do mesmo lado do vetor perpendicular a A
		double operator % (Point q) {return x * q.y - y * q.x;} //a%b = |a||b|sin(ang) //Angle of vectors
		double polar() { return ((y > -eps) ? atan2(y,x) : 2*pi + atan2(y,x)); }
		Point rotate(double t) { return Point(x * cos(t) - y * sin(t), x * sin(t) + y * cos(t)); }
		Point rotateAroundPoint(double t, Point p) {
			return (this - p).rotate(t) + p;
		}
		bool operator < (Point other) const {
			if(other.x != x) return x < other.x;
			else return y < other.y;
		}
	};
	
	struct Line {
		double a, b, c;
		Line(){};
		Line(double _a, double _b, double _c) {
			a = _a;
			b = _b;
			c = _c;
		}
		Line(Point s, Point t) {
			a = t.y - s.y;
			b = s.x - t.x;
			c = -a * s.x - b * s.y;
		}
		bool parallel(Line other) { return fabs(a * other.b - b * other.a) < eps; }
		Point intersect(Line other) {
			if(this->parallel(other)) return Point(-HUGE_VAL, -HUGE_VAL);
			else {
				double determinant = this->b * other.a - this->a * other.b;
				double x = (this->c * other.b - this->b * other.c) / determinant;
				double y = (this->a * other.c - this->c * other.a) / determinant;
				return Point(x, y);
			}
		}
		Line perpendicular(Point point) {
			return Line(-b, a, b * point.x - a * point.y);
		}
		double distance(Point r) {
			Point p, q;
			if(fabs(b) < eps) {
				p = Point(-c / a, 0);
				q = Point((-c - b) / a, 1);
			}
			else {
				p = Point(0, -c / b);
				q = Point(1, (-c - a) / b);
			}
			Point A = r - q, B = r - p, C = q - p;
			double a = A * A, b = B * B, c = C * C;
			return fabs(A % B) / sqrt(c);
		}
	};
	
	class GeometricUtils {
		public:
		GeometricUtils(){};
		static double cross(Point a, Point b, Point c) {
			double dx1 = (a.x - b.x), dy1 = (a.y - b.y);
			double dx2 = (c.x - b.x), dy2 = (c.y - b.y);
			return (dx1 * dy2 - dx2 * dy1);
		}
		static bool above(Point a, Point b, Point c) {
			return cross(a, b, c) < 0;
		}
		static bool under(Point a, Point b, Point c) {
			return cross(a, b, c) > 0;
		}
		static bool sameLine(Point a, Point b, Point c) {
			return cross(a, b, c) < eps;
		}
		static double segDistance(Point p, Point q, Point r) {
			Point A = r - q, B = r - p, C = q - p;
			double a = A * A, b = B * B, c = C * C;
			if (cmp(b, a + c) >= 0) return sqrt(a);
			else if (cmp(a, b + c) >= 0) return sqrt(b);
			else return fabs(A % B) / sqrt(c);
		}
	};
	
	struct Circle {
		double x, y, r;
		Circle(){};
		Circle(double _x, double _y, double _r) {
			x = _x;
			y = _y;
			r = _r;
		}
		Circle(Point a, Point b, Point c) {
			Line ab = Line(a, b);
			Line bc = Line(b, c);
			double xAB = (a.x + b.x) * 0.5;
			double yAB = (a.y + b.y) * 0.5;
			double xBC = (b.x + c.x) * 0.5;
			double yBC = (b.y + c.y) * 0.5;
			ab = ab.perpendicular(Point(xAB, yAB));
			bc = bc.perpendicular(Point(xBC, yBC));
			if(ab.parallel(bc)) {
				x = -1;
				y = -1;
				r = -1;
			}
			Point center = ab.intersect(bc);
			x = center.x;
			y = center.y;
			r = center.distance(a);
		}
		double getIntersectionArea(Circle c) {
			double d = hypot(x - c.x, y - c.y);
			if(d >= r + c.r) return 0.0;
			else if (c.r >= d + r) return pi * r * r;
			else if (r >= d + c.r) return pi * c.r * c.r;
			else {
				double a1 = 2.*acos((d * d + r * r - c.r * c.r) / (2. * d * r));
				double a2 = 2.*acos((d * d + c.r * c.r - r * r) / (2. * d * c.r));
				double num1 = (ld)a1 / 2. * r * r - r * r * sin(a1) * 0.5;
				double num2 = (ld)a2 / 2. * c.r * c.r - c.r * c.r * sin(a2)*0.5;
				return num1 + num2;
			}
		}
	};
	
	Point getCircuncenter(Point a, Point b, Point c) {
		Line l1 = Line(a, b);
		double xab = (a.x + b.x) * 0.5, yab = (a.y + b.y) * 0.5;
		Line l2 = Line(b, c);
		double xbc = (b.x + c.x) * 0.5, ybc = (b.y + c.y) * 0.5;
		l1 = l1.perpendicular(Point(xab, yab));
		l2 = l2.perpendicular(Point(xbc, ybc));
		return l1.intersect(l2);
	}
	
	vector< Point > ConvexHull(vector< Point > &polygon) {
		sort(polygon.begin(), polygon.end());
		vector< Point > down, up;
		up.pb(polygon[0]);
		up.pb(polygon[1]);
		down.pb(polygon[0]);
		down.pb(polygon[1]);
		for(int i = 2; i < polygon.size(); ++i) {
			while(up.size() >= 2 && GeometricUtils.above(up[up.size() - 2], up[up.size() - 1], polygon[i])) up.pop_back();
			while(down.size() >= 2 && GeometricUtils.under(down[down.size() - 2], down[down.size() - 1], polygon[i])) down.pop_back();
			up.pb(polygon[i]);
			down.pb(polygon[i]);
		}
		vector< Point > sol = up;
		for(int i = down.size() - 2; i > 0; --i) sol.pb(down[i]);
		return sol;
	}
	
	\end{lstlisting}
\end{multicols}

\section{Menor círculo}
Menor círculo que engloba todos os pontos. $O(n)$.
\begin{multicols}{2}
	\begin{lstlisting}
struct point{
	double x, y;
	point(){}
	point(double _x, double _y){
		x = _x, y = _y;
	}
	point subtract(point p){
		return point(x-p.x, y-p.y);
	}
	void read() { scanf("%lf %lf", &x, &y); }
	double distance(point p){
		return hypot(x-p.x, y-p.y);
	}
	double norm(){
		return x*x + y*y;
	}
	double cross(point p){
		return x*p.y - y*p.x;
	}
};
struct circle{
	double x, y, r;
	circle(){}
	circle(double _x, double _y, double _r){
		x = _x, y = _y, r = _r;
	}
	circle(point a, double b){
		x = a.x, y = a.y;
		r = b;
	}
	bool contains(point p){
		return point(x, y).distance(p) <= r + eps;
	}
	bool contains(vector<point> ps){
		for(auto p : ps)
		if(!contains(p))
		return 0;
		return 1;
	}
};
	
circle *makeCircumcircle(point a, point b, point c){
	double d = (a.x * (b.y - c.y) + b.x * (c.y - a.y) + c.x * (a.y - b.y)) * 2;
	if(d == 0)
	return NULL;
	double x = (a.norm() * (b.y - c.y) + b.norm() * (c.y - a.y) + c.norm() * (a.y - b.y)) / d;
	double y = (a.norm() * (c.x - b.x) + b.norm() * (a.x - c.x) + c.norm() * (b.x - a.x)) / d;
	point p = point(x, y);
	return new circle(p, p.distance(a));
}

circle makeDiameter(point a, point b){
	return circle(point((a.x + b.x)/ 2, (a.y + b.y) / 2), a.distance(b) / 2);
}

circle makeCircleTwoPoints(vector<point> points, point p, point q){
	circle temp = makeDiameter(p, q);
	if(temp.contains(points))
	return temp;
	
	circle *left = NULL;
	circle *right = NULL;
	
	for(point r : points){
		point pq = q.subtract(p);
		double cross = pq.cross(r.subtract(p));
		circle *c = makeCircumcircle(p, q, r);
		if(c == NULL)
		continue;
		else if(cross > 0 && (left == NULL || pq.cross(point(c->x, c->y).subtract(p)) > pq.cross(point(left->x, left->y).subtract(p))))
		left = c;
		else if (cross < 0 && (right == NULL || pq.cross(point(c->x, c->y).subtract(p)) < pq.cross(point(right->x, right->y).subtract(p))))
		right = c;
	}
	return right == NULL || left != NULL && left->r <= right->r ? *left : *right;
}

circle makeCircleOnePoint(vector<point> points, point p){
	circle c = circle(p, 0);
	for(int i = 0; i < points.size(); i++){
		point q = points[i];
		if(!c.contains(q)){
			if(c.r == 0)
			c = makeDiameter(p, q);
			else{
				vector<point> aux(&points[0], &points[i + 1]);
				c = makeCircleTwoPoints(aux, p, q);
			}
		}
	}
	return c;
}

circle makeCircle(vector<point> points){
	vector<point> shuffled = points;
	random_shuffle(shuffled.begin(), shuffled.end());
	
	circle c;
	bool first = true;
	for(int i = 0; i < shuffled.size(); i++){
		point p = shuffled[i];
		if(first || !c.contains(p))	{
			vector<point> aux(&shuffled[0], &shuffled[i + 1]);
			c = makeCircleOnePoint(aux, p);
			first = false;
		}
	}
	return c;
}
	\end{lstlisting}
\end{multicols}

\section{Kit de encolhimento - SBC 2016}

Encontra a menor área de um polígono convexo os vértices são deslocados ou para o ponto médio de Ax ou Bx.

\begin{multicols}{2}
	\begin{lstlisting}

#define inf 0x3f3f3f3f
#define eps 1e-9
#define MAXN 100010

struct point{
	double x, y;
	point(){}
	point(double a, double b){
		x = a;
		y = b;
	}
	point operator-(point other){
		return point(x-other.x, y-other.y);
	}
	point operator+(point other){
		return point(x+other.x, y+other.y);
	}
	point operator/(double v){
		return point(x/v, y/v);
	}
	double operator*(point q){
		return x*q.x + y*q.y;
	}
	double angle(){
		return atan2(double(y), double(x));
	}
	void read() { scanf("%lf %lf", &x, &y); }
};
double cross(point p, point q) { return p.x*q.y-p.y*q.x; }

// Sort by angle with pivot
bool cmp(point a, point b){
	point pivot = point(px, py);
	int x = direction(pivot, a, b);
	if(x == 0)
		return (pivot-a)*(pivot-a) < (pivot-b)*(pivot-b);
	return x==1;
}

double det(point a, point b, point c){
	return (a.x * b.y) + (b.x * c.y) + (c.x * a.y) - (a.x * c.y) - (b.x * a.y) - (c.x * b.y);
	double val = (b.y-a.y) * (c.x-b.x) - 
	(b.x-a.x) * (c.y-b.y);
	return val;
}

int direction(point a, point b, point c){
	double val = det(a, b, c);
	if(fabs(val) < eps)
	return 0;
	return val > 0 ? 1 : 2; // 0 Colinear, 1 Clockwise, 2 Counter
}

double area(point a, point b, point c){
	return fabs(det(a, b, c));
}

int n;
point v[MAXN];
point medio[MAXN][2];
point A, B;

double dp[MAXN][2][2][2][3];
int vis[MAXN][2][2][2][3];
int cnt;
int first, second;

double solve(int id, int penul, int ult, int temArea, int ori){
	if(id == n){
		int o1 = direction(medio[id-2][penul], medio[id-1][ult], medio[0][first]);
		int o2 = direction(medio[id-1][ult], medio[0][first], medio[1][second]);
		// Tratar concavidade na hora de fechar o poligono
		if(o2 == 0) 
		o2 = o1;
		if(o1 != o2) 
		return 1LL<<60;
		if(o1 != 0 && o1 != ori) 
		return 1LL<<60;
		if(temArea == 0) 
		return 1LL<<60;
		return 0;
	}
	double &ans = dp[id][penul][ult][temArea][ori];
	if(vis[id][penul][ult][temArea][ori] == cnt)
	return ans;
	vis[id][penul][ult][temArea][ori] = cnt;
	ans = 1LL<<60;
	for(int i = 0; i < 2; i++){
		int going = direction(medio[id-2][penul], medio[id-1][ult], medio[id][i]);
		double a = area(medio[0][first], medio[id-1][ult], medio[id][i]);
		
		if(ori == 0){
			ans = min(ans, a+solve(id+1, ult, i, temArea|(a>eps), going));
		}
		else if(going == ori || going == 0){
			// Tratar casos de espiral
			double c = cross(medio[id][i] - medio[id-1][ult], medio[0][first] - medio[id-1][ult]);
			if(going == 0)
			ans = min(ans, a+solve(id+1, ult, i, temArea|(a>eps), ori));
			else{
				if(fabs(c) < eps)
				ans = min(ans, a+solve(id+1, ult, i, temArea|(a>eps), ori));
				else if(going == 2 && c > 0) 
				continue;
				else if(going == 1 && c < 0) 
				continue;
				else 
				ans = min(ans, a+solve(id+1, ult, i, temArea|(a>eps), ori));
			}
			
			
		}
	}
	return ans;
}

int main(){
	scanf("%d", &n);
	for(int i = 0; i < n; i++)
	v[i].read();
	A.read();
	B.read();
	
	for(int i = 0; i < n; i++){
		medio[i][0] = (v[i]+A)/2;
		medio[i][1] = (v[i]+B)/2;			
	}
	
	double ans = 1LL<<60;
	for(int i = 0; i < 2; i++){
		first = i;
		for(int j = 0; j < 2; j++){
			second = j;
			cnt++;
			for(int k = 0; k < 2; k++){
				double a = area(medio[0][i], medio[1][j], medio[2][k]);
				ans = min(ans,  a+solve(3, j, k, a > eps, direction(medio[0][i], medio[1][j], medio[2][k])));
			}
		}
	}
	printf("%.3lf\n", ans/2.0);
	return 0;
}
	\end{lstlisting}
\end{multicols}

\section{Intersecção círculo e segmento}


\begin{multicols}{2}
	\begin{lstlisting}
_inline(int cmp)(double x, double y = 0, double tol = EPS)
{
	return (x <= y + tol) ? (x + tol < y) ? -1 : 0 : 1;
}
double abs(point p)
{
	return hypot(p.x, p.y);
}
double arg(point p)
{
	return atan2(p.y, p.x);
}
typedef pair<point, double> circle;

bool in_circle(circle C, point p)
{
	return cmp(abs(p - C.first), C.second) <= 0;
}

//////////////////////////////////////////////////////
// Encontra o ponto do segmento mais proximo a C
point circle_closest_point_seg(point p, point q, circle c)
{
	point A = q - p, B = c.first - p;
	double proj = (A * B)/abs(A);
	if (cmp(proj) < 0) return p;
	if (cmp(proj, abs(A)) > 0) return q;
	return p + (A * proj)/abs(A);
}
//////////////////////////////////////////////////////
// Decide se o segmento pq se interseca com c
bool seg_circle_intersect(point p, point q, circle c)
{
	point r = circle_closest_point_seg(p, q, c);
	return in_circle(c, r);
}
	\end{lstlisting}
\end{multicols}

