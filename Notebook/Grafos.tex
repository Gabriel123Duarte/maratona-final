\chapter{Grafos}

\section{Ford Fulkerson}

Encontra o fluxo máximo em $O(|\textit{f*}|E)$.
\begin{multicols}{2}
	\begin{lstlisting}
#define MAXN 100000
struct node{
	int v, f, c;
	node(){}
	node(int _v, int _f, int _c){
		v = _v, f = _f, c = _c;
	}
};

vector<node> edges;
vector<int> graph[MAXN];
int vis[MAXN];
int cnt;

void add(int u, int v, int c){
	edges.pb(node(v, 0, c));
	graph[u].pb(edges.size()-1);
	edges.pb(node(u, 0, 0));
	graph[v].pb(edges.size()-1);
}

int dfs(int s, int t, int f){
	if(s == t)
	return f;
	vis[s] = cnt;
	for(auto e : graph[s]){
		if(vis[edges[e].v] < cnt && edges[e].c-edges[e].f > 0){
			if(int x = dfs(edges[e].v, t, min(f,edges[e].c-edges[e].f))){
				edges[e].f += x;
				edges[e^1].f -= x;
				return x;
			}
		}
	}
	return 0;
}

int maxFlow(int s, int t){
	int ans = 0;
	cnt = 1;
	memset(vis, 0, sizeof vis);
	while(int flow = dfs(s, t, 1<<30)){
		ans += flow;
		cnt++;
	}
	return ans;
}
	\end{lstlisting}
\end{multicols}


\section{Edmonds Karp}

Troca a $dfs()$ do Ford Fulkerson por uma $bfs()$ e o fluxo máximo fica em $O(VE^2)$.

\section{Dinic}

Encontra o fluxo máximo em $O(V^2E)$.
\begin{multicols}{2}
	\begin{lstlisting}
#define MAXN 5050
#define inf 0x3f3f3f3f

struct node{
	int v, f, c;
	node(){}
	node(int _v, int _f, int _c){
		v = _v, f = _f, c = _c;
	}
};
vector<node> edges;
vector<int> graph[MAXN];
int dist[MAXN];
int ptr[MAXN];

void add(int u, int v, int c){
	edges.pb(node(v, 0, c));
	graph[u].pb(edges.size()-1);
	edges.pb(node(u, 0, 0));
	graph[v].pb(edges.size()-1);
}

bool bfs(int s, int t){
	memset(dist, inf, sizeof dist);
	dist[s] = 0;
	queue<int> q;
	q.push(s);
	
	while(!q.empty()){
		int u = q.front(); q.pop();
		for(auto e : graph[u]){
			if(dist[edges[e].v] == inf && edges[e].c-edges[e].f > 0){
				q.push(edges[e].v);
				dist[edges[e].v] = dist[u] + 1;
			}
		}
	}
	
	return dist[t] != inf;
}

int dfs(int s, int t, int f){
	if(s == t)
	return f;
	for(int &i = ptr[s]; i < graph[s].size(); i++){
		int e = graph[s][i];
		if(dist[edges[e].v] == dist[s]+1 && edges[e].c-edges[e].f > 0){
			if(int x = dfs(edges[e].v, t, min(f, edges[e].c-edges[e].f))){
				edges[e].f += x;
				edges[e^1].f -= x;
				return x;
			}
		}
	}
	
	return 0;
}

int maxFlow(int s, int t){
	int ans = 0;
	while(bfs(s, t)){
		memset(ptr, 0, sizeof ptr);
		while(int f = dfs(s, t, inf))
		ans += f;
	}
	return ans;
}
\end{lstlisting}
\end{multicols}

\section{Min cost max flow}

Máximo fluxo com custo mínimo.
\begin{multicols}{2}
	\begin{lstlisting}

#define MAXN 1100
#define inf 0x3f3f3f3f
struct node{
	int v, f, c, val;
	node(){}
	node(int _v, int _f, int _c, int _val){
		v = _v, f = _f, c = _c, val = _val;
	}
};

int v;
vector<node> edges;
vector<int> graph[MAXN];
int dist[MAXN], ptr[MAXN], pai[MAXN];

void add(int u, int v, int c, int val){
	edges.pb(node(v, 0, c, val));
	graph[u].pb(edges.size()-1);
	edges.pb(node(u, 0, 0, -val));
	graph[v].pb(edges.size()-1);
}

ii operator+(ii a, ii b){
	a.fst += b.fst;
	a.snd += b.snd;
	return a;
}

bool dijkstra(int s, int t){
	for(int i = 0; i < v; i++){
		dist[i] = inf;
		pai[i] = -1;
	}
	dist[s] = 0;
	priority_queue<ii, vector<ii>, greater<ii>> q;
	q.push(ii(0, s));
	
	while(!q.empty()){
		int d = q.top().fst, u = q.top().snd;
		q.pop();
		
		if(d > dist[u])
			continue;
		
		for(auto e : graph[u]){
			if(dist[u] + edges[e].val < dist[edges[e].v] && edges[e].c-edges[e].f > 0){
				dist[edges[e].v] = dist[u] + edges[e].val;
				pai[edges[e].v] = u;
				q.push({dist[edges[e].v], edges[e].v});
			}
		}
	}
	
	return dist[t] != inf;
}

ii dfs(int s, int t, int f){
	if(s == t)
		return ii(0, f);
	
	for(int &i = ptr[s]; i < graph[s].size(); i++){
		int e = graph[s][i];
		if(pai[edges[e].v] == s && dist[edges[e].v] == dist[s] + edges[e].val && edges[e].c-edges[e].f > 0){
			ii x = ii(edges[e].val, 0) + dfs(edges[e].v, t, min(f, edges[e].c-edges[e].f));
			if(x.snd){
				edges[e].f += x.snd;
				edges[e^1].f -= x.snd;
				return x;
			}
		}
	}
	
	return ii(0, 0);
}

ii get(int s, int t){
	ii ans(0, 0);
	while(dijkstra(s, t)){
		memset(ptr, 0, sizeof ptr);
		ii x;
		while((x = dfs(s, t, inf)).snd)
			ans = ans + x;
	}
	return ans;
}

\end{lstlisting}
\end{multicols}
\section{Stoer-Wagner}

Custo mínimo para quebrar o grafo em dois componentes.
\begin{multicols}{2}
	\begin{lstlisting}
	
#define NN 105 // Vertices
#define MAXW 105 // Max value of edge

int g[NN][NN], v[NN], w[NN], na[NN]; //graph comeca com tudo 0
bool a[NN];

int minCut(int n)
{
	for(int i = 0; i < n; i++)
	v[i] = i;
	
	int best = MAXW * n * n;
	while(n > 1)
	{
		a[v[0]] = true;
		for(int i = 1; i < n; i++)
		{
			a[v[i]] = false;
			na[i - 1] = i;
			w[i] = g[v[0]][v[i]];
		} 
		
		int prev = v[0];
		for(int i = 1; i < n; i++)
		{
			int zj = -1;
			for(int j = 1; j < n; j++)
			if(!a[v[j]] && (zj < 0 || w[j] > w[zj]))
			zj = j;
			
			a[v[zj]] = true;
			
			if(i == n - 1)
			{
				best = min(best, w[zj]);
				for(int j = 0; j < n; j++)
				g[v[j]][prev] = g[prev][v[j]] += g[v[zj]][v[j]];
				v[zj] = v[--n];
				break;
			}
			prev = v[zj];	
			
			for(int j = 1; j < n; j++ )
			if(!a[v[j]])
			w[j] += g[v[zj]][v[j]];
		}	
	}
	return best;
}
\end{lstlisting}
\end{multicols}
\section{Tarjan}

Componentes fortemente conexos em $O(V+E)$.
\begin{multicols}{2}
	\begin{lstlisting}
#define MAXN 100100
vector<int> graph[MAXN];
stack<int> st;
int in[MAXN], low[MAXN], vis[MAXN], cnt;
int sccs;

void dfs(int u){
	in[u] = low[u] = cnt++;
	vis[u] = 1;
	st.push(u);
	
	for(auto v : graph[u]){
		if(!vis[v]){
			dfs(v);
			low[u] = min(low[u], low[v]);
		}
		else
			low[u] = min(low[u], in[v]);
	}
	if(low[u] == in[u]){
		sccs++;
		int x;
		do{
			x = st.top();
			st.pop();
			in[x] = 1<<30;
		}while(x != u);
	}
}

void tarjan(int n){
	cnt = sccs = 0;
	memset(vis, 0, sizeof vis);
	while(!st.empty())
		st.pop();
	for(int i = 0; i < n; i++)
		if(!vis[i])
			dfs(i);
}
\end{lstlisting}
\end{multicols}

\section{Pontos de articulação}

Complexidade $O(V+E)$.
\begin{multicols}{2}
	\begin{lstlisting}
#define MAXN 100100
vector<int> graph[MAXN];
int in[MAXN], low[MAXN], vis[MAXN], cnt;
vector<int> points;

void dfs(int u int root){
	in[u] = low[u] = cnt++;
	vis[u] = 1;
	int total = 0;	
	bool ok = 0;
	for(auto v : graph[u]){
		if(!vis[v]){
			dfs(v, root);
			low[u] = min(low[u], low[v]);
			total++;
			if(low[v] >= in[u])
				ok = 1;
			// if(low[v] > in[u]) u-v eh uma ponte
		}
		else
			low[u] = min(low[u], in[v]);
		
	}
	if(u == root && total >= 2 || ok && u != root)
		points.pb(u);
}

void getPoints(int n){
	cnt = 0;
	points.clear();
	memset(vis, 0, sizeof vis);
	for(int i = 0; i < n; i++)
		if(!vis[i])
			dfs(i, i);
}
\end{lstlisting}
\end{multicols}

\section{LCA (Sparce Table)}

Complexidade $<O(nlog), O(log)>$.
\begin{multicols}{2}
	\begin{lstlisting}
#define INF 0x3f3f3f3f
#define N 1100
#define LOG 15

int parents[N][LOG], depth[N];
vector<vi> graph;

void dfs(int u, int p, int h){
	parents[u][0] = p;
	depth[u] = h;
	
	for (int i = 1; i < LOG; i++)
	if (parents[u][i - 1] != -1)
	parents[u][i] = parents[parents[u][i - 1]][i - 1];
	
	for (int i = 0; i < graph[u].size(); i++)
	if (graph[u][i] != parents[u][0])
	dfs(graph[u][i], u, h + 1);	
}

int lca(int u, int v){
	if (depth[u] > depth[v])
	swap(u, v);
	
	for (int i = LOG - 1; i >= 0; i--)
	if (parents[v][i] != -1 && depth[parents[v][i]] >= depth[u])
	v = parents[v][i];
	
	if (u == v) 
	return u;
	
	for (int i = LOG - 1; i >= 0; i--)
	if (parents[u][i] != parents[v][i]){
		u = parents[u][i];
		v = parents[v][i];
	}
	
	return parents[u][0];	
}
\end{lstlisting}
\end{multicols}

\section{Posição de elemento em K passos em um ciclo}

Encontra onde estará um elemento após executar K passos dentro de um ciclo $O(n log n)$

\begin{multicols}{2}
	\begin{lstlisting}
	
#define N 100000
#define LOG 31
int dp[N][LOG];

// Caso base
for(int i = 0; i < n; i++)
	dp[i][0] = ligacao[i];

for(int i = 1; i < LOG; i++)
	for(int j = 0; j < n; j++)
		dp[j][i] = dp[dp[j][i-1]][i-1];
		
for(int j = 0; j < LOG; j++)
	if(k&(1<<j))
		u = dp[u][j]; 
\end{lstlisting}
\end{multicols}

\section{Hopcroft Karp}

Maior matching em grafo bipartido.q	

\begin{multicols}{2}
	\begin{lstlisting}
#define MAXN 100100
vector<int> graph[MAXN];
int dist[MAXN], match[MAXN];

bool bfs(){
	queue<int> q;
	for(int i = 1; i <= n; i++){
		if(!match[i]){
			dist[i] = 0;
			q.push(i);
		}
		else
		dist[i] = 1<<30;
	}
	
	dist[0] = 1<<30;
	while(!q.empty()){
		int u = q.front(); q.pop();
		if(u){
			for(auto v : graph[u])
			if(dist[match[v]] == 1<<30){
				dist[match[v]] = dist[u]+1;
				q.push(match[v]);
			}
		}
	}
	return dist[0] != 1<<30;
}

bool dfs(int u){
	if(u){
		for(auto v : graph[u]){
			if(dist[match[v]] == dist[u]+1){
				if(dfs(match[v])){
					match[v] = u;
					match[u] = v;
					return 1;
				}
			}
		}
		dist[u] = 1<<30;
		return 0;
	}
	return 1;
}
// Grafo indexado de 1
int hopcroftKarp(){
	int ans = 0;
	memset(match, 0, sizeof match);
	while(bfs())
		for(int i = 1; i <= n; i++)
			if(!match[i] && dfs(i))
				ans++;
	return ans;
}

\end{lstlisting}
\end{multicols}

\section{Blossom}
Matching para grafos genéricos.
\begin{multicols}{2}
	\begin{lstlisting}

int lca(vi &match, vi &base, vi &p, int a, int b){
	vi used(match.size(), 0);
	while(1){
		a = base[a];
		used[a] = 1;
		if(match[a] == -1)
			break;
		a = p[match[a]];
	}
	while(1){
		b = base[b];
		if(used[b])
			return b;
		b = p[match[b]];
	}
}

void markPath(vi &match, vi &base, vi &blossom, vi &p, int v, int b, int children){
	for(; base[v] != b; v = p[match[v]]){
		blossom[base[v]] = blossom[base[match[v]]] = true;
		p[v] = children;
		children = match[v];
	}
}

int findPath(vector<vi> &graph, vi &match, vi &p, int root){
	int n = graph.size();
	vi used(n, 0);
	fill(p.begin(), p.end(), -1);
	vi base(n, 0);
	for(int i = 0; i < n; i++)
		base[i] = i;
	
	used[root] = 1;
	int qh = 0, qt = 0;
	
	vi q(n, 0);
	q[qt++] = root;
	while(qh < qt){
		int v = q[qh++];
		for(int to : graph[v]){
			if(base[v] == base[to] || match[v] == to)
				continue;
			if(to == root || match[to] != -1 && p[match[to]] != -1)	{
				int curbase = lca(match, base, p, v, to);
				vi blossom(n, 0);
				markPath(match, base, blossom, p, v, curbase, to);
				markPath(match, base, blossom, p, to, curbase, v);
				
				for(int i = 0; i < n; i++){
					if(blossom[base[i]]){
						base[i] = curbase;
						if(!used[i]){
							used[i] = true;
							q[qt++] = i;
						}
					}
				}
			}
			else if(p[to] == -1){
				p[to] = v;
				if(match[to] == -1)
					return to;
				to = match[to];
				used[to] = true;
				q[qt++] = to;
			}
		}
	}
	return -1;
}

int maxMatching(vector<vi> &graph){
	int n = graph.size();
	vi match(n, -1);
	vi p(n, 0);
	for(int i = 0; i < n; i++){
		if(match[i] == -1){
			int v = findPath(graph, match, p, i);
			while(v != -1){
				int pv = p[v];
				int ppv = match[pv];
				match[v] = pv;
				match[pv] = v;
				v = ppv;
			}
		}
	}
	
	int matches = 0;
	for(int i = 0; i < n; i++)
		if(match[i] != -1)
			matches++;
	//printf("Aqui %d\n", matches);
	return matches;
}
\end{lstlisting}
\end{multicols}