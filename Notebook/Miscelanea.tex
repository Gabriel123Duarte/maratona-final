\chapter{Miscelânea}

\section{Datas}

\begin{multicols}{2}
	\begin{lstlisting}

struct Date {
	int d, m, y;
	static int mnt[], mntsum[];
	
	Date() : d(1), m(1), y(1) {}
	Date(int d, int m, int y) : d(d), m(m), y(y) {}
	Date(int days) : d(1), m(1), y(1) { advance(days); }
	
	bool bissexto() { return (y%4 == 0 and y%100) or (y%400 == 0); }
		
	int mdays() { return mnt[m] + (m == 2)*bissexto(); }
	int ydays() { return 365+bissexto(); }
		
	int msum()  { return mntsum[m-1] + (m > 2)*bissexto(); }
	int ysum()  { return 365*(y-1) + (y-1)/4 - (y-1)/100 + (y-1)/400; }
		
	int count() { return (d-1) + msum() + ysum(); }
		
	int day() {
		int x = y - (m<3);
		return (x + x/4 - x/100 + x/400 + mntsum[m-1] + d + 6)%7;
	}
		
	void advance(int days) {
		days += count();
		d = m = 1, y = 1 + days/366;
		days -= count();
		while(days >= ydays()) days -= ydays(), y++;
		while(days >= mdays()) days -= mdays(), m++;
		d += days;
	}
};
	
int Date::mnt[13] = {0, 31, 28, 31, 30, 31, 30, 31, 31, 30, 31, 30, 31};
int Date::mntsum[13] = {};
for(int i=1; i<13; ++i) Date::mntsum[i] = Date::mntsum[i-1] + Date::mnt[i];

// Week day
int v[] = { 0, 3, 2, 5, 0, 3, 5, 1, 4, 6, 2, 4 };
int day(int d, int m, int y) {
	y -= m<3;
	return (y + y/4 - y/100 + y/400 + v[m-1] + d)%7;
}
\end{lstlisting}
\end{multicols}

\section{Hash C++11}
\begin{lstlisting}
hash<string> hashFunc;
cout << hashFunc("Gabriel") << endl;
\end{lstlisting}

\section{Mo's}
Cada query em ~$O(sqrt(N)$.
\begin{multicols}{2}
\begin{lstlisting}

#define MAXN 100100
int BLOCK; // ~sqrt(MAXN)

struct query{
	int l, r, id;
	bool operator<(const query foo)const{
		if(l / BLOCK != foo.l / BLOCK)
			return l / BLOCK < foo.l / BLOCK;
		return r < foo.r;
	}
};

int vis[MAXN], resp;

void add(int id){
	if(!vis[id]){
		// Add element and update resp
	}
	else{
		// Remove element and update resp
	}
	vis[id] ^= 1;
}

// IN MAIN
int q;
sort(queries, queries+q);
int ans[q];
int curL = 0, curR = 0;
resp = 0;

for(int i = 0; i < q; i++){
	int l = queries[i].l, r = queries[i].r;
	
	while(curL < l)	add(curL++);
	while(curL > l) add(--curL);
	while(curR <= r) add(curR++);
	while(curR > r+1) add(--curR--);
	ans[queries[i].id] = resp;
}

\end{lstlisting}
\end{multicols}

\section{Problema do histograma}
Maior retângulo do histograma em $O(n)$.
\begin{multicols}{2}
	\begin{lstlisting}

ll histograma(ll vet[], int n){
	stack<ll> s;
	ll ans = 0, tp, cur;
	int i = 0;
	while(i < n || !s.empty()){
		if(i < n && (s.empty() || vet[s.top()] <= vet[i]))
			s.push(i++);
		else{
			tp = s.top();
			s.pop();
			cur = vet[tp]*(s.empty() ? i : i - s.top() - 1);
			if(ans < cur)
				ans = cur;
		}
	}
	return ans;
}

\end{lstlisting}
\end{multicols}
