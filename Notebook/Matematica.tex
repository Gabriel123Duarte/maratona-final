\chapter{Matemática}


\section{Eliminação de Gauss com o XOR}

Retorna o valor máximo de xor que é possível se obter fazendo xor entre os elementos da array. \\	
Pode ser necessário o ull ou bitset.
\begin{multicols}{2}
	\begin{lstlisting}

int len(ll x){
	int ans = 0;
	while(x){
		ans++;
		x >>= 1;
	}
	return ans;
}
	
ll gaussxor(ll arr[], int n){
	vector<ll> buckets[65];
	for(int i = 0; i < n; i++)
		buckets[len(arr[i])].pb(arr[i]);
	
	vector<ll> modified;
	for(int i = 64; i; i--){
		if(buckets[i].size()){
			modified.pb(buckets[i][0]);
			for(int j = 1; j < buckets[i].size(); j++){
				ll temp = buckets[i][0] ^ buckets[i][j];
				buckets[len(temp)].pb(temp);
			}
		}
	}
	
	// Ans = maximum xor subset
	ll ans = 0;
	for(auto m : modified)
		if(ans < ans^m)
			ans ^= m;
	return ans;
}
	\end{lstlisting}
\end{multicols}
\section{Fórmula de Legendre}

Dados um inteiro $n$ e um primo $p$, calcula o expoente da maior potência de $p$ que divide $n!$ em $O(logn)$.
\begin{multicols}{2}
	\begin{lstlisting}
ll legendre(ll n, ll p){
	ll ans = 0;
	ll prod = p;
	while(prod <= n){
		ans += n/prod;
		prod *= p;
	}
	return ans;
}
	\end{lstlisting}
\end{multicols}
