\chapter{Matemática}


\section{Eliminação de Gauss com o XOR}

Retorna o valor máximo de xor que é possível se obter fazendo xor entre os elementos da array. \\	
Pode ser necessário o ull ou bitset.
\begin{multicols}{2}
	\begin{lstlisting}

int len(ll x){
	int ans = 0;
	while(x){
		ans++;
		x >>= 1;
	}
	return ans;
}
	
ll gaussxor(ll arr[], int n){
	vector<ll> buckets[65];
	for(int i = 0; i < n; i++)
		buckets[len(arr[i])].pb(arr[i]);
	
	vector<ll> modified;
	for(int i = 64; i; i--){
		if(buckets[i].size()){
			modified.pb(buckets[i][0]);
			for(int j = 1; j < buckets[i].size(); j++){
				ll temp = buckets[i][0] ^ buckets[i][j];
				buckets[len(temp)].pb(temp);
			}
		}
	}
	
	// Ans = maximum xor subset
	ll ans = 0;
	for(auto m : modified)
		if(ans < ans^m)
			ans ^= m;
	return ans;
}
	\end{lstlisting}
\end{multicols}
\section{Fórmula de Legendre}

Dados um inteiro $n$ e um primo $p$, calcula o expoente da maior potência de $p$ que divide $n!$ em $O(logn)$.
\begin{multicols}{2}
	\begin{lstlisting}
ll legendre(ll n, ll p){
	ll ans = 0;
	ll prod = p;
	while(prod <= n){
		ans += n/prod;
		prod *= p;
	}
	return ans;
}
	\end{lstlisting}
\end{multicols}

\section{Número de fatores primos de N!}

Dado um N encontra quantos fatores o N! possui
\begin{multicols}{2}
	\begin{lstlisting}
// Sieve of Eratosthenes to mark all prime number
// in array prime as 1
void sieve(int n, bool prime[])
{
	// Initialize all numbers as prime
	for (int i=1; i<=n; i++)
	prime[i] = 1;
	
	// Mark composites
	prime[1] = 0;
	for (int i=2; i*i<=n; i++)
	{
		if (prime[i])
		{
			for (int j=i*i; j<=n; j += i)
			prime[j] = 0;
		}
	}
}

// Returns the highest exponent of p in n!
int expFactor(int n, int p)
{
	int x = p;
	int exponent = 0;
	while ((n/x) > 0)
	{
		exponent += n/x;
		x *= p;
	}
	return exponent;
}

// Returns the no of factors in n!
ll countFactors(int n)
{
	// ans stores the no of factors in n!
	ll ans = 1;
	
	// Find all primes upto n
	bool prime[n+1];
	sieve(n, prime);
	
	// Multiply exponent (of primes) added with 1
	for (int p=1; p<=n; p++)
	{
		// if p is a prime then p is also a
		// prime factor of n!
		if (prime[p]==1)
		ans *= (expFactor(n, p) + 1);
	}
	
	return ans;
}

\end{lstlisting}
\end{multicols}

\section{Trailing zeros in factorial}

\begin{multicols}{2}
	\begin{lstlisting}
int findTrailingZeros(int  n){
	// Initialize result
	int count = 0;
	
	// Keep dividing n by powers of 5 and update count
	for (int i=5; n/i>=1; i *= 5)
		count += n/i;
	
	return count;
}
\end{lstlisting}
\end{multicols}


\section{Número de divisores de N!}

Dado um N encontra quantos divisores o N! possui
\begin{multicols}{2}
	\begin{lstlisting}

// allPrimes[] stores all prime numbers less
// than or equal to n.
vector<ull> allPrimes;

// Fills above vector allPrimes[] for a given n
void sieve(int n)
{
	// Create a boolean array "prime[0..n]" and
	// initialize all entries it as true. A value
	// in prime[i] will finally be false if i is
	// not a prime, else true.
	vector<bool> prime(n+1, true);
	
	// Loop to update prime[]
	for (int p=2; p*p<=n; p++)
	{
		// If prime[p] is not changed, then it
		// is a prime
		if (prime[p] == true)
		{
			// Update all multiples of p
			for (int i=p*2; i<=n; i += p)
			prime[i] = false;
		}
	}
	
	// Store primes in the vector allPrimes
	for (int p=2; p<=n; p++)
	if (prime[p])
	allPrimes.push_back(p);
}

// Function to find all result of factorial number
ull factorialDivisors(ull n)
{
	sieve(n);  // create sieve
	
	// Initialize result
	ull result = 1;
	
	// find exponents of all primes which divides n
	// and less than n
	for (int i=0; i < allPrimes.size(); i++)
	{
		// Current divisor
		ull p = allPrimes[i];
		
		// Find the highest power (stored in exp)'
		// of allPrimes[i] that divides n using
		// Legendre's formula.
		ull exp = 0;
		while (p <= n)
		{
			exp = exp + (n/p);
			p = p*allPrimes[i];
		}
		
		// Multiply exponents of all primes less
		// than n
		result = result*(exp+1);
	}
	
	// return total divisors
	return result;
}
\end{lstlisting}
\end{multicols}

\section{Grundy Number}
Faz o xor de todos os números grundy de todas as pilhas, se for diferente de 0 ganha o jogo.
\begin{multicols}{2}
	\begin{lstlisting}

ll mex(unordered_set<ll> st){
	ll ans = 0;
	while(st.count(ans)) ans++;
	return ans;
}

ll grundy(int x){
	if(x perde jogo)
		return 0;
		
	unordered_set<ll> st;
	for(int i = 0; i < l; i++)
		st.insert(grundy(novoX)); // Transicoes
	return mex(st);
}
\end{lstlisting}
\end{multicols}

\section{Baby-Step Giant-Step para Logaritmo Discreto}
Resolve a equação $a^x = b(modm)$ em $O(sqrt(m)log m)$. Retorna -1 se não há solução.
\begin{multicols}{2}
\begin{lstlisting}
template <typename T>
T baby (T a, T b, T m) {
	a %= m; b %= m;
	T n = (T) sqrt (m + .0) + 1;
	T an = 1;
	for (T i=0; i<n; ++i)
		an = (an * a) % m;
	map<T,T> vals;
	for (T i=1, cur=an; i<=n; ++i) {
		if (!vals.count(cur))
			vals[cur] = i;
		cur = (cur * an) % m;
	}
	for (T i=0, cur=b; i<=n; ++i) {
		if (vals.count(cur)) {
			T ans = vals[cur] * n - i;
			if (ans < m)
				return ans;
			}
			cur = (cur * a) % m;
	}
	return -1;
}

\end{lstlisting}
\end{multicols}
\section{Baby-Step Giant-Step FAST}
Resolve a equação $a^x = b(modm)$ em $O(sqrt(m)log m)$. Retorna -1 se não há solução.
\begin{multicols}{2}
	\begin{lstlisting}
class Hash
{
	static const int HASHMOD = 7679977;
	int top, myhash[HASHMOD + 100], value[HASHMOD + 100], stack[1 << 16];
	int locate(const int x) const
	{
		int h = x % HASHMOD;
		while (myhash[h] != -1 && myhash[h] != x) ++h;
		return h;
	}
	public:
	Hash() : top(0) { memset(myhash, 0xFF, sizeof(myhash)); }
	void insert(const int x, const int v)
	{
		const int h = locate(x);
		if (myhash[h] == -1)
		myhash[h] = x, value[h] = v, stack[++top] = h;
	}
	int get(const int x) const
	{
		const int h = locate(x);
		return myhash[h] == x ? value[h] : -1;
	}
	void clear() { while (top) myhash[stack[top--]] = -1; }
} myhash;

struct Triple
{
	ll x, y, z;
	Triple(const ll a, const ll b, const ll c) : x(a), y(b), z(c) { }
};

Triple ExtendedGCD(const ll a, const ll b)
{
	if (b == 0) return Triple(1, 0, a);
	const Triple last = ExtendedGCD(b, a%b);
	return Triple(last.y, last.x - a / b * last.y, last.z);
}

ll BabyStep(ll A, ll B, ll C)
{
	const int sqrtn = static_cast<int>(std::ceil(std::sqrt(C)));
	ll base = 1;
	myhash.clear();
	for (int i = 0; i < sqrtn; ++i)
	{
		myhash.insert(base, i);
		base = base*A % C;
	}
	ll i = 0, j = -1, D = 1;
	for (; i < sqrtn; ++i)
	{
		Triple res = ExtendedGCD(D, C);
		const int c = C / res.z;
		res.x = (res.x * B / res.z % c + c) % c;
		j = myhash.get(res.x);
		if (j != -1) return i * sqrtn + j;
		D = D * base % C;
	}
	return -1;
}


\end{lstlisting}
\end{multicols}

\section{Números de Catalan}
Computa os números de Catalan de 0 até $n$ em $(nlogn)$.\\
Olhar mais exemplos no CP3 pg. 206
\begin{itemize}
\itemsep0em
\item $Cat(n)$ = número de árvores binárias completas de $n$+1 folhas ou 2*$n$ + 1 elementos
\item $Cat(n)$ = número de combinações válidas para $n$ pares de parêntesis.
\item $Cat(n)$ = número de formas que o parentesiação de $n$+1 elementos pode ser feito.
\item $Cat(n)$ = número de triangulações de um polígono convexo de $n$+2 lados.
\item $Cat(n)$ = número de caminhos monotônicos discretos para ir de (0,0) a (n,n).
\end{itemize}

\begin{multicols}{2}
\begin{lstlisting}

ll cat[MAXN];
void catalan(int n){
	cat[0] = cat[1] = 1;
	ll g;
	for(int i = 2; i<=n; i++){
		g = gcd(2*(2*i-1), i+1);
		cat[i] = ((2*(2*i-1))/g)*(cat[i-1]/((i+1)/g));
	}
}	

\end{lstlisting}
\end{multicols}

\section{Fórmulas úteis}
Olhar mais fórmulas no CP3 pg. 345
\begin{itemize}
\itemsep0em
\item Soma dos n primeiros fibonacci: $f(n+2)-1$.
\item Soma dos n primeiros fibonacci ao quadrado: $f(n)*f(n+1)$.
\item Soma dos quadrados de todos números de 1 até n:   $n*(n+1)*(2n+1)/6$.
\item Fórmula de Cayley: existem $n^{n-2}$ árvores geradoras em um grafo completo de $n$ vértices.
\item Desarranjo: o número $der(n)$ de permutações de $n$ elementos em que nenhum dos elementos fica na posição original é dado por:
$der(n) = (n-1)(der(n-1)+der(n-2))$, onde $der(0) = 1$ e $der(1) = 0$.
\item Teorema de Erdos Gallai: é condição suficiente para que uma array represente os graus dos vértices de um nó: $d_1 \geq d_2 \geq ... \geq d_n$, $\sum_{i=1}^{n}d_i = 2k$, $\sum_{i=1}^{k}d_i \leq k(k-1) + \sum_{i=k+1}^{k}min(d_i, k)$.
\item Fórmula de Euler para grafos planares: $V-E+F=2$, onde $F$ é o número de faces.
\item Círculo de Moser: o número de peças em que um círculo pode ser divido por cordas ligadas a $n$ pontos tais que não se tem 3 cordas internamente concorrentes é dada por: $g(n) = C_{4}^{n}+C_{2}^{n}+1$.
\item Teorema de Pick: se $I$ é o número de pontos inteiros dentro de um polígono, $A$ a área do polígono e $b$ o número de pontos inteiros na borda, então $A = i + b/2 - 1$ .
\item O número de árvores geradores em um grafo bipartido completo é $m^{n-1} \times n^{m-1}$.
\item Teorema de Kirchhoff: o número de árvores geradoras em um grafo é igual ao cofator da sua matriz laplaciana $L$. $L=D-A$, em que $D$ é uma matriz diagonal em que $a_{ii} = d_i$ e $A$ é a matriz de adjacência.
\item Teorema de Konig: a cobertura mínima de vértices em um grafo bipartido (o número mínimo de vértices a serem removidos para se remover todas as arestas) é igual ao pareamento máximo do grafo.
\item Teorema de Zeckendorf: qualquer inteiro positivo pode ser representado pela soma de números de Fibonacci que não inclua dois números consecutivos. Para achar essa soma, usar o algoritmo guloso, sempre procurando o maior número de fibonacci menor que o número.
\item Teorema de Dilworth: em um DAG que representa um conjunto parcialmente ordenado, uma cadeia é um subconjunto de vértices tais que todos os pares dentro dele são comparáveis; uma anti-cadeia é um subconjunto tal que todos os pares de vértices dele são não comparáveis. O teorema afirma que a partição mínima em cadeias é igual ao comprimenton da maior anti-cadeia. Para computar, criar um grafo bipartido: para cada vértice $x$, duplicar para $u_x$ e $v_x$. Uma aresta $x \rightarrow y$ é escrita como $u_x \rightarrow v_y$. O tamanho da partição mínima, também chamada de largura do conjunto, é $N-$ o emparelhamento máximo.
\item Teorema de Mirsky: semelhante ao teorema de Dilworth, o tamanho da partição mínima em anti-cadeias é igual ao comprimento da maior cadeia.
\end{itemize}


\section{Fast Fourier Transform - FFT}

\begin{multicols}{2}
	\begin{lstlisting}

struct base {
	double x, y;
	base() : x(0), y(0) {}
	base(double a, double b=0) : x(a), y(b) {}
	base operator/=(double k) { x/=k; y/=k; return (*this); }
	base operator*(base a) const { return base(x*a.x - y*a.y, x*a.y + y*a.x); }
	base operator*=(base a) {
		double tx = x*a.x - y*a.y;
		double ty = x*a.y + y*a.x;
		x = tx; y = ty;
		return (*this);
	}
	base operator+(base a) const { return base(x+a.x, y+a.y); }
	base operator-(base a) const { return base(x-a.x, y-a.y); }
	double real() { return x; }
	double imag() { return y; }
};

typedef complex<double> base; // Se tiver com tempo bom

void fft (vector<base> & a, bool invert) {
	int n = (int)a.size();
	for (int i=1, j=0; i<n; ++i) {
		int bit = n >> 1;
		for (; j>=bit; bit>>=1)
			j -= bit;
		j += bit;
		if (i < j) swap(a[i], a[j]);
	}
	
	for (int len=2; len<=n; len<<=1) {
		double ang = 2*M_PI/len * (invert ? -1 : 1);
		base wlen(cos(ang), sin(ang));
		for (int i=0; i<n; i+=len) {
			base w(1);
			for (int j=0; j<len/2; ++j) {
				base u = a[i+j],  v = a[i+j+len/2] * w;
				a[i+j] = u + v;
				a[i+j+len/2] = u - v;
				w *= wlen;
			}
		}
	}
	if (invert)
	for (int i=0; i<n; ++i)
	a[i] /= n;
}

void convolution (vector<base> a, vector<base> b, vector<base> & res) {
	int n = 1;
	while(n < max(a.size(), b.size())) n <<= 1;
	n <<= 1;
	a.resize(n), b.resize(n);
	fft(a, false); fft(b, false);
	res.resize(n);
	for(int i=0; i<n; ++i) res[i] = a[i]*b[i];
	fft(res, true);
}

\end{lstlisting}
\end{multicols}

\section{Convolução Circular}
\begin{lstlisting}

template <typename T>
void circularconvolution(vector<T> a, vector<T> b, vector<T> & res) {
	int n = a.size();
	b.insert(b.end(), b.begin(), b.end());
	convolution(a, b, res);
	res = vector<T>(res.begin()+n, res.begin()+(2*n));
}

\end{lstlisting}

