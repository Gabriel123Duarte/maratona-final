\chapter{Paradigmas}


\section{Convex hull trick 1}

Quando o X está ordenado. \\
Inserir retas do tipo Y = A*X + B. \\
Para máximo adicionar o A e B negativos e quando consultar coloca negativo o valor.

\begin{multicols}{2}
\begin{lstlisting}
struct hull{
	ll A[MAXN];
	ll B[MAXN];
	int len, ptr;
	hull(){
		len = ptr = 0;
	}

	void addLine(ll a, ll b){
		while(len >= 2 && (B[len-2] - B[len-1]) * (a-A[len-1]) >= (B[len-1]-b) * (A[len-1]-A[len-2]))
			len--;
		A[len] = a;
		B[len] = b;
		len++;
	}

	ll get(ll x){
		ptr = min(ptr, len-1);
		while(ptr+1 < len && A[ptr+1]*x+B[ptr+1] <= A[ptr]*x + B[ptr])
			ptr++;
		return A[ptr]*x + B[ptr];
	}
};
\end{lstlisting}
\end{multicols}

\section{Convex hull trick 2}

Quando o X não está ordenado. 

\begin{multicols}{2}
\begin{lstlisting}
class ConvexHullTrick {
	struct CHTPoint {
		double x, y, lim;
	};
	vector<CHTPoint> pilha;
	inline double get_intersection(CHTPoint a, CHTPoint b) {
		double denom = ( double(b.x) - a.x);
		double num = ( double(b.y) - a.y);

		return -num / denom;
	}
	
	bool ccw(CHTPoint p0, CHTPoint p1, CHTPoint p2) {
		return ((double)(p1.y-p0.y)*(double)(p2.x-p0.x) > (double)(p2.y-p0.y)*(double)(p1.x-p0.x));
	}
	
	public:
	
	void add_line(double a, double b) {
		CHTPoint novo = {a, b, 0};
		int tam = pilha.size();
		while (tam >= 2 && !ccw(pilha[tam-2], pilha[tam-1], novo)) {
			pilha.pop_back();
			tam--;
		}
		while (tam >= 1 && fabs(pilha[tam-1].x - a) < 1e-8) {
			pilha.pop_back();
			tam--;
		}
	
		pilha.push_back(novo);
		if (tam >= 1) pilha[tam-1].lim = get_intersection(pilha[tam-1], pilha[tam]);
	}
	
	double get_maximum(double x) {
		int st = 0, ed = pilha.size() - 1;
		while (st < ed) {
			int mid = (st+ed)/2;
			if (pilha[mid].lim < x) st = mid+1;
			else ed = mid;
		}
		return pilha[st].x * x + pilha[st].y;
	}
	
};
\end{lstlisting}
\end{multicols}

\section{Convex hull trick 3 - Fully Dynamic}

Sem condições especiais para o A e B

\begin{multicols}{2}
\begin{lstlisting}
const ll is_query = -(1LL<<62);
struct Line {
	ll m, b;
	mutable function<const Line*()> succ;
	bool operator<(const Line& rhs) const {
		if (rhs.b != is_query) return m < rhs.m;
		const Line* s = succ();
		if (!s) return 0;
		ll x = rhs.m;
		return b - s->b < (s->m - m) * x;
	}
};

struct HullDynamic : public multiset<Line> { // will maintain upper hull for maximum
	bool bad(iterator y) {
		auto z = next(y);
		if (y == begin()) {
			if (z == end()) return 0;
			return y->m == z->m && y->b <= z->b;
		}
		auto x = prev(y);
		if (z == end()) return y->m == x->m && y->b <= x->b;
		return (x->b - y->b)*(z->m - y->m) >= (y->b - z->b)*(y->m - x->m);
	}
	void insert_line(ll m, ll b) {
		auto y = insert({ m, b });
		y->succ = [=] { return next(y) == end() ? 0 : &*next(y); };
		if (bad(y)) { erase(y); return; }
		while (next(y) != end() && bad(next(y))) erase(next(y));
		while (y != begin() && bad(prev(y))) erase(prev(y));
	}
	ll eval(ll x) {
		auto l = *lower_bound((Line) { x, is_query });
		return l.m * x + l.b;
	}
};
\end{lstlisting}
\end{multicols}

\section{Divide and conquer optimization}

Dividir elementos em K pilhas sai de $O(KN^2)$ para $O(KNlogN)$.

\begin{multicols}{2}
\begin{lstlisting}
void solve(int i, int left, int right, int kleft, int kright){
	if(left > right)
	return;
	
	int jmid = (right+left)/2;
	dp[i][jmid] = 1<<30;
	int bestk = -1;
	for(int k = kleft; k <= min(kright, jmid); k++){
		if(dp[i-1][k] + get(k+1, jmid) < dp[i][jmid]){
			dp[i][jmid] = dp[i-1][k] + get(k+1, jmid); // Cost function
			bestk = k;
		}
	}
	
	solve(i, left, jmid-1, kleft, bestk);
	solve(i, jmid+1, right, bestk, kright);		
}

// Chamada fica assim
for(int i = 1; i <= n; i++) // Casos base
	dp[1][i] = get(1, i);

for(int i = 2; i <= k; i++)
	solve(i, 1, n, 1, n);
\end{lstlisting}
\end{multicols}

\section{Knuth optimization}

Dividir elementos P elementos em A pilhas.

\begin{multicols}{2}
\begin{lstlisting}
for(int i = 0; i <= p + 1; i++)
	for(int j = 0; j <= a + 1; j++)
		dpAux[i][j] = 1;
	
for(int i = 1; i <= p; i++)
	dp[i][1] = prefix[i] * i;

for(int i = 0; i <= a; i++)
	dpAux[p + 1][i] = p;

for(int j = 2; j <= a; j++)
	for(int i = p; i >= 1; i--){
		dp[i][j] = 1LL << 61;
		for(int x = dpAux[i][j - 1]; x <= dpAux[i + 1][j]; x++){
			if(dp[i][j] >= dp[x][j - 1] + get(x + 1, i) * (i - x)){
				dp[i][j] = dp[x][j - 1] + get(x + 1, i) * (i - x);
				dpAux[i][j] = x;					
			}
		}
	}

\end{lstlisting}
\end{multicols}

\section{Otimização com bitmask}

$dp[i]$ é igual a um bitmask onde o $i$-ésimo bit informa se é possível fazer a soma $i$ usando $x$ elementos, onde $x$ é a posição do bitmask.

\begin{multicols}{2}
\begin{lstlisting}
dp[0] = 1;
for(int i = 0; i < n; i++)
	for(int j = sum/2; j >= v[i]; j--)
		dp[j] |= dp[j-v[i]]<<1LL;

int ans = sum/2;
if(n&1){
	ll x = 1LL<<(n/2);
	ll y = 1LL<<(n/2+1);
	while(!(dp[ans]&x) && !(dp[ans]&y))
		ans--;
}
else
	while(!(dp[ans]&(1LL<<(n/2))))	
		ans--;
\end{lstlisting}
\end{multicols}
