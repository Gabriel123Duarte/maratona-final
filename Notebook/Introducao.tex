\chapter{Introdução}

\section{Template}

Digitar o template no inicio da prova. \\
\textbf{NÃO} esquecer de remover o $freopen()$
\begin{multicols}{2}
\begin{lstlisting}
#include <bits/stdc++.h>
using namespace std;
#define all(v) (v).begin(), (v).end()
#define pb push_back
#define fst first
#define snd second
#define debug(x) cout << #x << " = " << x << endl;
typedef long long ll;
typedef pair<int, int> ii;

int main(){
	freopen("in", "rt", stdin);

	return 0;
}
\end{lstlisting}
\end{multicols}

\section{Fast Input}

Em casos extremos mete isso sem medo. 
\begin{multicols}{2}
\begin{lstlisting}
template<class num>inline void rd(num &x)
{
	char c;
	while(isspace(c = getchar()));
	bool neg = false;
	if(!isdigit(c))
		neg=(c=='-'), x=0;
	else
		x=c-'0';
	while(isdigit(c=getchar()))
		x=(x<<3)+(x<<1)+c-'0';
	if(neg)
		x=-x;
}
\end{lstlisting}
\end{multicols}

\section{Bugs do Milênio}

Cortesia da ITA. \\

\begin{multicols}{2}

\textbf{Erros teóricos:}
\begin{itemize}
\itemsep0em
\item Não ler o enunciado do problema com calma.
\item Assumir algum fato sobre a solução na pressa.
\item Não reler os limites do problema antes de submeter.
\item Quando adaptar um algoritmo, atentar para todos os detalhes da estrutura do algoritmo, se devem (ou não) ser modificados (ex:marcação de vértices/estados).
\item O problema pode ser NP, disfarçado ou mesmo sem limites especificados. Nesse caso a solução é bronca mesmo. Não é hora de tentar ganhar o prêmio nobel.
\end{itemize}

\textbf{Erros com valor máximo de variável:}
\begin{itemize}
\itemsep0em
\item Verificar com calma (fazer as contas direito) para ver se o infinito é tão infinito quanto parece. 
\item Verificar se operações com infinito estouram 31 bits.
\item Usar multiplicação de $int$'s e estourar 32 bits (por exemplo, checar sinais usando $a*b > 0$).
\end{itemize}

\textbf{Erros de casos extremos:}
\begin{itemize}
\itemsep0em
\item Testou caso $n=0$? $n=1$? $n=MAXN$? Muitas vezes tem que tratar separado.
\item Pense em todos os casos que podem ser considerados casos extremos ou casos isolados.
\item Casos extremos podem atrapalhar não só no algoritmo, mas em coisas como construir alguma estrutura (ex: lista de adj em grafos).
\item Não esquecer de self-loops ou multiarestas em grafos.
\item Em problemas de caminho Euleriano, verificar se o grafo é conexo.
\end{itemize}

\textbf{Erros de desatenção em implementação:}
\begin{itemize}
\itemsep0em
\item Errar ctrl-C/ctrl-V em código. Muito comum.
\item Colocar igualdade dentro de $if$? ($if(a = 0) continue;$)
\item Esquecer de inicializar variável.
\item Trocar $break$ por $continue$ (ou vice-versa).
\item Declarar variável global e variável local com mesmo nome (é pedir pra dar merda...).
\end{itemize}

\textbf{Erros de implementação:}
\begin{itemize}
\itemsep0em
\item Definir variável com tipo errado ($int$ por $double$, $int$ por $char$).
\item Não usar variável com nome $max$ e $min$.
\item Não esquecer que $.size()$ é unsigned.
\item Lembrar que 1 é $int$, ou seja, se fizer $long \; long \; a \; = \; 1 \; << \; 40;$, não irá funcionar (o ideal é fazer $long \; long \; a \; = \; 1LL \; << \; 40;$).
\end{itemize}

\textbf{Erros em limites:}
\begin{itemize}
\itemsep0em
\item Qual o ordem do tempo e memória? $10^8$ é uma referência para tempo. Sempre verificar rapidamente a memória, apesar de que o limite costuma ser bem grande.
\item A constante pode ser muito diminuída com um algoritmo melhor (ex: húngaro no lugar de fluxo) ou com operações mais rápidas (ex: divisões são lentas, bitwise é rápido)?
\item O exercício é um caso particular que pode (e está precisando) ser otimizado e não usar direto a biblioteca? 
\end{itemize}


\textbf{Erros em doubles:}
\begin{itemize}
\itemsep0em
\item Primeiro, evitar (a não ser que seja necessário ou mais simples a solução) usar $float/double$. E.g. conta que só precisa de 2 casas decimais pode ser feita com inteiro e depois $\% 100$.
\item Sempre usar $double$, não $float$ (a não ser que o enunciado peça explicitamente).
\item Testar igualdade com tolerância (absoluta, e talvez relativa).
\item Cuidado com erros de imprecisão, em particular evitar ao máximo subtrair dois números praticamente iguais.
\end{itemize}

\textbf{Outros erros:}
\begin{itemize}
\itemsep0em
\item Evitar (a não ser que seja necessário) alocação dinâmica de memória.
\item Não usar STL desnecessariamente (ex: vector quando um array normal dá na mesma), mas usar se facilitar (ex: nomes associados a vértices de um grafo - $map<string,int>$) ou se precisar (ex: um algoritmo $O(nlogn)$ que usa $<set>$ é necessário para passar no tempo).
\item Não inicializar variável a cada teste (zerou vetores? zerou variável que soma algo? zerou com zero? era pra zerar com zero, com -1 ou com INF?).
\item Saída está formatada corretamente?
\item Declarou vetor com tamanho suficiente?
\item Cuidado ao tirar o módulo de número negativo. Ex.: $x\%n$ não dá o resultado esperado se x é negativo, fazer $(x\%n + n)\%n$.
\end{itemize}

\end{multicols}

\section{Recomendações gerais}

Cortesia da PUC-RJ. \\

\textbf{ANTES DA PROVA}
\begin{itemize}
\itemsep0em
\item Revisar os algoritmos disponíveis na biblioteca.
\item Revisar a referência STL.
\item Reler este roteiro.
\item Ouvir o discurso motivacional do técnico.
\end{itemize}

\textbf{ANTES DE IMPLEMENTAR UM PROBLEMA}
\begin{itemize}
\itemsep0em
\item Quem for implementar deve relê-lo antes.
\item Peça todas as clarifications que forem necessárias.
\item Marque as restrições e faça contas com os limites da entrada.
\item Teste o algoritmo no papel e convença outra pessoa de que ele funciona.
\item Planeje a resolução para os problemas grandes: a equipe se junta para definir as estruturas de dados, mas cada pessoa escreve uma função.
\end{itemize}

\textbf{DEBUGAR UM PROGRAMA}
\begin{itemize}
\itemsep0em
\item Ao encontrar um bug, escreva um caso de teste que o dispare.
\item Reimplementar trechos de programas entendidos errados.
\item Em caso de RE, procure todos os [, / e \%.
\end{itemize}

\section{Os 1010 mandamentos}

Também cortesia da PUC-RJ.
\\\\
0. Não dividirás por zero.\\
1. Não alocarás dinamicamente.\\
2. Compararás números de ponto flutuante usando EPS.\\
3. Verificarás se o grafo pode ser desconexo.\\
4. Verificarás se as arestas do grafo podem ter peso negativo.\\
5. Verificarás se pode haver mais de uma aresta ligando dois vértices.\\
6. Conferirás todos os índices de uma programação dinâmica.\\
7. Reduzirás o branching factor da DFS.\\
8. Farás todos os cortes possíveis em uma DFS.\\
9. Tomarás cuidado com pontos coincidentes e com pontos colineares.\\

\newpage

\section{Limites da representação de dados}

\begin{table}[h!]
\centering{
\begin{tabular}{|c|c|ccc|c|}
\hline
tipo & bits & mínimo &..& máximo & precisão decimal\\
\hline

char & $8 $	& $0$ &..& $127 $ & $2 $\\
signed char & $8 $& $-128$ &..& $127 	$ & $2 $\\
unsigned char & $8 $ & $0$ &..& $255 $ & $2 $\\
short & $16$ & $-32.768$ &..& $32.767 $ & $4 $\\
unsigned short & $16$ & $0$ &..& $65.535 $ & $4 $\\
int & $32$ 	& $-2 \times 10^9$ &..& $2 \times 10^9 $ & $9 $\\
unsigned int & $32$ & $0$ &..& $4 \times 10^9 $ & $9 $\\
long long & $64$ & $-9 \times 10^{18}$ &..& $9 \times 10^{18} $ & $18$ \\
unsigned long long & $64$ 	& $0$ &..& $18 \times 10^{18} 				$ & $19$ \\

\hline

\end{tabular}
}
\end{table}

\begin{table}[h!]
\centering{
\begin{tabular}{|c|c|c|c|}
\hline
tipo & bits & expoente & precisão decimal \\
\hline
float & 32 & 38 & 6 \\
double & 64 & 308 & 15 \\
long double & 80 & 19.728 & 18 \\

\hline

\end{tabular}
}
\end{table}

\section{Quantidade de números primos de $1$ até $10^n$}

É sempre verdade que $n / ln(n) < pi(n) < 1.26 * n / ln(n)$.

\begin{table}[h!]
\centering{
\begin{tabular}{|c|c|c|}
\hline
$pi(10^1) = 4$ &
$pi(10^2) = 25$ &
$pi(10^3) = 168$ \\
$pi(10^4) = 1.229$ &
$pi(10^5) = 9.592$ &
$pi(10^6) = 78.498$ \\
$pi(10^7) = 664.579$ &
$pi(10^8) = 5.761.455$ &
$pi(10^9) = 50.847.534$ \\

\hline

\end{tabular}
}
\end{table}

\section{Triângulo de Pascal}

\begin{table}[h!]
\centering{
\begin{tabular}{|c|ccccccccccc|}
\hline
$n \backslash p$ & $0$ & $1$ & $2$ & $3$ & $4$ & $5$ & $6$ & $7$ & $8$ & $9$ & $10$ \\
\hline
$0 $ & $1$ &&&&&&&&&&\\
$1 $ & $1$ & $1 $ &&&&&&&&&\\
$2 $ & $1$ & $2 $ & $1 $ &&&&&&&&\\
$3 $ & $1$ & $3 $ & $3 $ & $1$   &&&&&&&\\
$4 $ & $1$ & $4 $ & $6 $ & $4$   & $1$  &&&&&&\\
$5 $ & $1$ & $5 $ & $10$ & $10$  & $5$  & $1$   &&&&&\\
$6 $ & $1$ & $6 $ & $15$ & $20$  & $15$ & $6$   & $1$   &&&&\\
$7 $ & $1$ & $7 $ & $21$ & $35$  & $35$ & $21$  & $7$   & $1$   &&&\\
$8 $ & $1$ & $8 $ & $28$ & $56$  & $70$ & $56$  & $28$  & $8$   & $1$  &&\\
$9 $ & $1$ & $9 $ & $36$ & $84$  & $126$& $126$ & $84$  & $36$  & $9$  & $1$  &\\
$10$ & $1$ & $10$ & $45$ & $120$ & $210$& $252$ & $210$ & $120$ & $45$ & $10$ & $1$\\

\hline

\end{tabular}
}
\end{table}

\begin{table}[h!]
\centering{
\begin{tabular}{|c|c|c|}
\hline
$C(33, 16)$ & $1.166.803.110$ & limite do int \\
$C(34, 17)$ & $2.333.606.220$ & limite do unsigned int \\
$C(66, 33)$ & $7.219.428.434.016.265.740$ & limite do long long \\
$C(67, 33)$ & $14.226.520.737.620.288.370$ & limite do unsigned long long \\
\hline
\end{tabular}
}
\end{table}

\newpage

\section{Fatoriais}

Fatoriais até 20 com os limites de tipo.

\begin{table}[h!]
\centering{
\begin{tabular}{|c|c|c|}
\hline
$0! $ & $1$ & \\
$1! $ & $1$ & \\
$2! $ & $2$ & \\
$3! $ & $6$ & \\
$4! $ & $24$ & \\
$5! $ & $120$ & \\
$6! $ & $720$ & \\
$7! $ & $5.040$ & \\
$8! $ & $40.320$ & \\
$9! $ & $362.880$ & \\
$10!$ & $3.628.800$ & \\
$11!$ & $39.916.800$ & \\
$12!$ & $479.001.600$ & limite do unsigned int \\
$13!$ & $6.227.020.800$ & \\
$14!$ & $87.178.291.200$ & \\
$15!$ & $1.307.674.368.000$ & \\
$16!$ & $20.922.789.888.000$ & \\
$17!$ & $355.687.428.096.000$ & \\
$18!$ & $6.402.373.705.728.000$ & \\
$19!$ & $121.645.100.408.832.000$ & \\
$20!$ & $2.432.902.008.176.640.000$ & limite do unsigned long long \\
\hline

\end{tabular}
}
\end{table}

\section{Tabela ASCII}

\begin{figure}[h]
	\centering
	\includegraphics[width=0.68125\textwidth]{ascii.jpg}
\end{figure}

\newpage

\section{Primos até 10.000}

Existem 1.229 números primos até 10.000.

\begin{table}[h!]
\centering{
\begin{tabular}{|c|c|c|c|c|c|c|c|c|c|c|}
\hline
$2   $ & $3   $ & $5   $ & $7   $ & $11  $ & $13  $ & $17  $ & $19  $ & $23  $ & $29  $ & $31  $ \\
$37  $ & $41  $ & $43  $ & $47  $ & $53  $ & $59  $ & $61  $ & $67  $ & $71  $ & $73  $ & $79  $ \\
$83  $ & $89  $ & $97  $ & $101 $ & $103 $ & $107 $ & $109 $ & $113 $ & $127 $ & $131 $ & $137 $ \\
$139 $ & $149 $ & $151 $ & $157 $ & $163 $ & $167 $ & $173 $ & $179 $ & $181 $ & $191 $ & $193 $ \\
$197 $ & $199 $ & $211 $ & $223 $ & $227 $ & $229 $ & $233 $ & $239 $ & $241 $ & $251 $ & $257 $ \\
$263 $ & $269 $ & $271 $ & $277 $ & $281 $ & $283 $ & $293 $ & $307 $ & $311 $ & $313 $ & $317 $ \\
$331 $ & $337 $ & $347 $ & $349 $ & $353 $ & $359 $ & $367 $ & $373 $ & $379 $ & $383 $ & $389 $ \\
$397 $ & $401 $ & $409 $ & $419 $ & $421 $ & $431 $ & $433 $ & $439 $ & $443 $ & $449 $ & $457 $ \\
$461 $ & $463 $ & $467 $ & $479 $ & $487 $ & $491 $ & $499 $ & $503 $ & $509 $ & $521 $ & $523 $ \\
$541 $ & $547 $ & $557 $ & $563 $ & $569 $ & $571 $ & $577 $ & $587 $ & $593 $ & $599 $ & $601 $ \\
$607 $ & $613 $ & $617 $ & $619 $ & $631 $ & $641 $ & $643 $ & $647 $ & $653 $ & $659 $ & $661 $ \\
$673 $ & $677 $ & $683 $ & $691 $ & $701 $ & $709 $ & $719 $ & $727 $ & $733 $ & $739 $ & $743 $ \\
$751 $ & $757 $ & $761 $ & $769 $ & $773 $ & $787 $ & $797 $ & $809 $ & $811 $ & $821 $ & $823 $ \\
$827 $ & $829 $ & $839 $ & $853 $ & $857 $ & $859 $ & $863 $ & $877 $ & $881 $ & $883 $ & $887 $ \\
$907 $ & $911 $ & $919 $ & $929 $ & $937 $ & $941 $ & $947 $ & $953 $ & $967 $ & $971 $ & $977 $ \\
$983 $ & $991 $ & $997 $ & $1009$ & $1013$ & $1019$ & $1021$ & $1031$ & $1033$ & $1039$ & $1049$ \\
$1051$ & $1061$ & $1063$ & $1069$ & $1087$ & $1091$ & $1093$ & $1097$ & $1103$ & $1109$ & $1117$ \\
$1123$ & $1129$ & $1151$ & $1153$ & $1163$ & $1171$ & $1181$ & $1187$ & $1193$ & $1201$ & $1213$ \\
$1217$ & $1223$ & $1229$ & $1231$ & $1237$ & $1249$ & $1259$ & $1277$ & $1279$ & $1283$ & $1289$ \\
$1291$ & $1297$ & $1301$ & $1303$ & $1307$ & $1319$ & $1321$ & $1327$ & $1361$ & $1367$ & $1373$ \\
$1381$ & $1399$ & $1409$ & $1423$ & $1427$ & $1429$ & $1433$ & $1439$ & $1447$ & $1451$ & $1453$ \\
$1459$ & $1471$ & $1481$ & $1483$ & $1487$ & $1489$ & $1493$ & $1499$ & $1511$ & $1523$ & $1531$ \\
$1543$ & $1549$ & $1553$ & $1559$ & $1567$ & $1571$ & $1579$ & $1583$ & $1597$ & $1601$ & $1607$ \\
$1609$ & $1613$ & $1619$ & $1621$ & $1627$ & $1637$ & $1657$ & $1663$ & $1667$ & $1669$ & $1693$ \\
$1697$ & $1699$ & $1709$ & $1721$ & $1723$ & $1733$ & $1741$ & $1747$ & $1753$ & $1759$ & $1777$ \\
$1783$ & $1787$ & $1789$ & $1801$ & $1811$ & $1823$ & $1831$ & $1847$ & $1861$ & $1867$ & $1871$ \\
$1873$ & $1877$ & $1879$ & $1889$ & $1901$ & $1907$ & $1913$ & $1931$ & $1933$ & $1949$ & $1951$ \\
$1973$ & $1979$ & $1987$ & $1993$ & $1997$ & $1999$ & $2003$ & $2011$ & $2017$ & $2027$ & $2029$ \\
$2039$ & $2053$ & $2063$ & $2069$ & $2081$ & $2083$ & $2087$ & $2089$ & $2099$ & $2111$ & $2113$ \\
$2129$ & $2131$ & $2137$ & $2141$ & $2143$ & $2153$ & $2161$ & $2179$ & $2203$ & $2207$ & $2213$ \\
$2221$ & $2237$ & $2239$ & $2243$ & $2251$ & $2267$ & $2269$ & $2273$ & $2281$ & $2287$ & $2293$ \\
$2297$ & $2309$ & $2311$ & $2333$ & $2339$ & $2341$ & $2347$ & $2351$ & $2357$ & $2371$ & $2377$ \\
$2381$ & $2383$ & $2389$ & $2393$ & $2399$ & $2411$ & $2417$ & $2423$ & $2437$ & $2441$ & $2447$ \\
$2459$ & $2467$ & $2473$ & $2477$ & $2503$ & $2521$ & $2531$ & $2539$ & $2543$ & $2549$ & $2551$ \\
$2557$ & $2579$ & $2591$ & $2593$ & $2609$ & $2617$ & $2621$ & $2633$ & $2647$ & $2657$ & $2659$ \\
$2663$ & $2671$ & $2677$ & $2683$ & $2687$ & $2689$ & $2693$ & $2699$ & $2707$ & $2711$ & $2713$ \\
$2719$ & $2729$ & $2731$ & $2741$ & $2749$ & $2753$ & $2767$ & $2777$ & $2789$ & $2791$ & $2797$ \\
$2801$ & $2803$ & $2819$ & $2833$ & $2837$ & $2843$ & $2851$ & $2857$ & $2861$ & $2879$ & $2887$ \\
$2897$ & $2903$ & $2909$ & $2917$ & $2927$ & $2939$ & $2953$ & $2957$ & $2963$ & $2969$ & $2971$ \\
$2999$ & $3001$ & $3011$ & $3019$ & $3023$ & $3037$ & $3041$ & $3049$ & $3061$ & $3067$ & $3079$ \\
$3083$ & $3089$ & $3109$ & $3119$ & $3121$ & $3137$ & $3163$ & $3167$ & $3169$ & $3181$ & $3187$ \\
$3191$ & $3203$ & $3209$ & $3217$ & $3221$ & $3229$ & $3251$ & $3253$ & $3257$ & $3259$ & $3271$ \\
$3299$ & $3301$ & $3307$ & $3313$ & $3319$ & $3323$ & $3329$ & $3331$ & $3343$ & $3347$ & $3359$ \\
$3361$ & $3371$ & $3373$ & $3389$ & $3391$ & $3407$ & $3413$ & $3433$ & $3449$ & $3457$ & $3461$ \\
$3463$ & $3467$ & $3469$ & $3491$ & $3499$ & $3511$ & $3517$ & $3527$ & $3529$ & $3533$ & $3539$ \\
$3541$ & $3547$ & $3557$ & $3559$ & $3571$ & $3581$ & $3583$ & $3593$ & $3607$ & $3613$ & $3617$ \\
$3623$ & $3631$ & $3637$ & $3643$ & $3659$ & $3671$ & $3673$ & $3677$ & $3691$ & $3697$ & $3701$ \\
$3709$ & $3719$ & $3727$ & $3733$ & $3739$ & $3761$ & $3767$ & $3769$ & $3779$ & $3793$ & $3797$ \\
$3803$ & $3821$ & $3823$ & $3833$ & $3847$ & $3851$ & $3853$ & $3863$ & $3877$ & $3881$ & $3889$ \\
$3907$ & $3911$ & $3917$ & $3919$ & $3923$ & $3929$ & $3931$ & $3943$ & $3947$ & $3967$ & $3989$ \\
$4001$ & $4003$ & $4007$ & $4013$ & $4019$ & $4021$ & $4027$ & $4049$ & $4051$ & $4057$ & $4073$ \\
$4079$ & $4091$ & $4093$ & $4099$ & $4111$ & $4127$ & $4129$ & $4133$ & $4139$ & $4153$ & $4157$ \\
\hline
\end{tabular}
}
\end{table}

\begin{table}[h!]
\centering{
\begin{tabular}{|c|c|c|c|c|c|c|c|c|c|c|}
\hline
$4159$ & $4177$ & $4201$ & $4211$ & $4217$ & $4219$ & $4229$ & $4231$ & $4241$ & $4243$ & $4253$ \\
$4259$ & $4261$ & $4271$ & $4273$ & $4283$ & $4289$ & $4297$ & $4327$ & $4337$ & $4339$ & $4349$ \\
$4357$ & $4363$ & $4373$ & $4391$ & $4397$ & $4409$ & $4421$ & $4423$ & $4441$ & $4447$ & $4451$ \\
$4457$ & $4463$ & $4481$ & $4483$ & $4493$ & $4507$ & $4513$ & $4517$ & $4519$ & $4523$ & $4547$ \\
$4549$ & $4561$ & $4567$ & $4583$ & $4591$ & $4597$ & $4603$ & $4621$ & $4637$ & $4639$ & $4643$ \\
$4649$ & $4651$ & $4657$ & $4663$ & $4673$ & $4679$ & $4691$ & $4703$ & $4721$ & $4723$ & $4729$ \\
$4733$ & $4751$ & $4759$ & $4783$ & $4787$ & $4789$ & $4793$ & $4799$ & $4801$ & $4813$ & $4817$ \\
$4831$ & $4861$ & $4871$ & $4877$ & $4889$ & $4903$ & $4909$ & $4919$ & $4931$ & $4933$ & $4937$ \\
$4943$ & $4951$ & $4957$ & $4967$ & $4969$ & $4973$ & $4987$ & $4993$ & $4999$ & $5003$ & $5009$ \\
$5011$ & $5021$ & $5023$ & $5039$ & $5051$ & $5059$ & $5077$ & $5081$ & $5087$ & $5099$ & $5101$ \\
$5107$ & $5113$ & $5119$ & $5147$ & $5153$ & $5167$ & $5171$ & $5179$ & $5189$ & $5197$ & $5209$ \\
$5227$ & $5231$ & $5233$ & $5237$ & $5261$ & $5273$ & $5279$ & $5281$ & $5297$ & $5303$ & $5309$ \\
$5323$ & $5333$ & $5347$ & $5351$ & $5381$ & $5387$ & $5393$ & $5399$ & $5407$ & $5413$ & $5417$ \\
$5419$ & $5431$ & $5437$ & $5441$ & $5443$ & $5449$ & $5471$ & $5477$ & $5479$ & $5483$ & $5501$ \\
$5503$ & $5507$ & $5519$ & $5521$ & $5527$ & $5531$ & $5557$ & $5563$ & $5569$ & $5573$ & $5581$ \\
$5591$ & $5623$ & $5639$ & $5641$ & $5647$ & $5651$ & $5653$ & $5657$ & $5659$ & $5669$ & $5683$ \\
$5689$ & $5693$ & $5701$ & $5711$ & $5717$ & $5737$ & $5741$ & $5743$ & $5749$ & $5779$ & $5783$ \\
$5791$ & $5801$ & $5807$ & $5813$ & $5821$ & $5827$ & $5839$ & $5843$ & $5849$ & $5851$ & $5857$ \\
$5861$ & $5867$ & $5869$ & $5879$ & $5881$ & $5897$ & $5903$ & $5923$ & $5927$ & $5939$ & $5953$ \\
$5981$ & $5987$ & $6007$ & $6011$ & $6029$ & $6037$ & $6043$ & $6047$ & $6053$ & $6067$ & $6073$ \\
$6079$ & $6089$ & $6091$ & $6101$ & $6113$ & $6121$ & $6131$ & $6133$ & $6143$ & $6151$ & $6163$ \\
$6173$ & $6197$ & $6199$ & $6203$ & $6211$ & $6217$ & $6221$ & $6229$ & $6247$ & $6257$ & $6263$ \\
$6269$ & $6271$ & $6277$ & $6287$ & $6299$ & $6301$ & $6311$ & $6317$ & $6323$ & $6329$ & $6337$ \\
$6343$ & $6353$ & $6359$ & $6361$ & $6367$ & $6373$ & $6379$ & $6389$ & $6397$ & $6421$ & $6427$ \\
$6449$ & $6451$ & $6469$ & $6473$ & $6481$ & $6491$ & $6521$ & $6529$ & $6547$ & $6551$ & $6553$ \\
$6563$ & $6569$ & $6571$ & $6577$ & $6581$ & $6599$ & $6607$ & $6619$ & $6637$ & $6653$ & $6659$ \\
$6661$ & $6673$ & $6679$ & $6689$ & $6691$ & $6701$ & $6703$ & $6709$ & $6719$ & $6733$ & $6737$ \\
$6761$ & $6763$ & $6779$ & $6781$ & $6791$ & $6793$ & $6803$ & $6823$ & $6827$ & $6829$ & $6833$ \\
$6841$ & $6857$ & $6863$ & $6869$ & $6871$ & $6883$ & $6899$ & $6907$ & $6911$ & $6917$ & $6947$ \\
$6949$ & $6959$ & $6961$ & $6967$ & $6971$ & $6977$ & $6983$ & $6991$ & $6997$ & $7001$ & $7013$ \\
$7019$ & $7027$ & $7039$ & $7043$ & $7057$ & $7069$ & $7079$ & $7103$ & $7109$ & $7121$ & $7127$ \\
$7129$ & $7151$ & $7159$ & $7177$ & $7187$ & $7193$ & $7207$ & $7211$ & $7213$ & $7219$ & $7229$ \\
$7237$ & $7243$ & $7247$ & $7253$ & $7283$ & $7297$ & $7307$ & $7309$ & $7321$ & $7331$ & $7333$ \\
$7349$ & $7351$ & $7369$ & $7393$ & $7411$ & $7417$ & $7433$ & $7451$ & $7457$ & $7459$ & $7477$ \\
$7481$ & $7487$ & $7489$ & $7499$ & $7507$ & $7517$ & $7523$ & $7529$ & $7537$ & $7541$ & $7547$ \\
$7549$ & $7559$ & $7561$ & $7573$ & $7577$ & $7583$ & $7589$ & $7591$ & $7603$ & $7607$ & $7621$ \\
$7639$ & $7643$ & $7649$ & $7669$ & $7673$ & $7681$ & $7687$ & $7691$ & $7699$ & $7703$ & $7717$ \\
$7723$ & $7727$ & $7741$ & $7753$ & $7757$ & $7759$ & $7789$ & $7793$ & $7817$ & $7823$ & $7829$ \\
$7841$ & $7853$ & $7867$ & $7873$ & $7877$ & $7879$ & $7883$ & $7901$ & $7907$ & $7919$ & $7927$ \\
$7933$ & $7937$ & $7949$ & $7951$ & $7963$ & $7993$ & $8009$ & $8011$ & $8017$ & $8039$ & $8053$ \\
$8059$ & $8069$ & $8081$ & $8087$ & $8089$ & $8093$ & $8101$ & $8111$ & $8117$ & $8123$ & $8147$ \\
$8161$ & $8167$ & $8171$ & $8179$ & $8191$ & $8209$ & $8219$ & $8221$ & $8231$ & $8233$ & $8237$ \\
$8243$ & $8263$ & $8269$ & $8273$ & $8287$ & $8291$ & $8293$ & $8297$ & $8311$ & $8317$ & $8329$ \\
$8353$ & $8363$ & $8369$ & $8377$ & $8387$ & $8389$ & $8419$ & $8423$ & $8429$ & $8431$ & $8443$ \\
$8447$ & $8461$ & $8467$ & $8501$ & $8513$ & $8521$ & $8527$ & $8537$ & $8539$ & $8543$ & $8563$ \\
$8573$ & $8581$ & $8597$ & $8599$ & $8609$ & $8623$ & $8627$ & $8629$ & $8641$ & $8647$ & $8663$ \\
$8669$ & $8677$ & $8681$ & $8689$ & $8693$ & $8699$ & $8707$ & $8713$ & $8719$ & $8731$ & $8737$ \\
$8741$ & $8747$ & $8753$ & $8761$ & $8779$ & $8783$ & $8803$ & $8807$ & $8819$ & $8821$ & $8831$ \\
$8837$ & $8839$ & $8849$ & $8861$ & $8863$ & $8867$ & $8887$ & $8893$ & $8923$ & $8929$ & $8933$ \\
$8941$ & $8951$ & $8963$ & $8969$ & $8971$ & $8999$ & $9001$ & $9007$ & $9011$ & $9013$ & $9029$ \\
$9041$ & $9043$ & $9049$ & $9059$ & $9067$ & $9091$ & $9103$ & $9109$ & $9127$ & $9133$ & $9137$ \\
$9151$ & $9157$ & $9161$ & $9173$ & $9181$ & $9187$ & $9199$ & $9203$ & $9209$ & $9221$ & $9227$ \\
$9239$ & $9241$ & $9257$ & $9277$ & $9281$ & $9283$ & $9293$ & $9311$ & $9319$ & $9323$ & $9337$ \\
$9341$ & $9343$ & $9349$ & $9371$ & $9377$ & $9391$ & $9397$ & $9403$ & $9413$ & $9419$ & $9421$ \\
$9431$ & $9433$ & $9437$ & $9439$ & $9461$ & $9463$ & $9467$ & $9473$ & $9479$ & $9491$ & $9497$ \\
$9511$ & $9521$ & $9533$ & $9539$ & $9547$ & $9551$ & $9587$ & $9601$ & $9613$ & $9619$ & $9623$ \\
$9629$ & $9631$ & $9643$ & $9649$ & $9661$ & $9677$ & $9679$ & $9689$ & $9697$ & $9719$ & $9721$ \\
$9733$ & $9739$ & $9743$ & $9749$ & $9767$ & $9769$ & $9781$ & $9787$ & $9791$ & $9803$ & $9811$ \\
$9817$ & $9829$ & $9833$ & $9839$ & $9851$ & $9857$ & $9859$ & $9871$ & $9883$ & $9887$ & $9901$ \\
$9907$ & $9923$ & $9929$ & $9931$ & $9941$ & $9949$ & $9967$ & $9973$ &        &        &       \\
\hline

\end{tabular}
}
\end{table}
