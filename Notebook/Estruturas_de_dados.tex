\chapter{Estruturas de dados}

\section{Prefix Sum 1D}

Soma $a..b$ em $O(1)$.
\begin{multicols}{2}
\begin{lstlisting}
#define MAXN 1000
int arr[MAXN];
int prefix[MAXN];

void build(int n){
	prefix[0] = 0;
	for(int i = 1; i <= n; i++) // arr 1-indexado
		prefix[i] = prefix[i-1]+arr[i];
}

int get(int a, int b){
	return prefix[b] - prefix[a-1];
}

\end{lstlisting}
\end{multicols}


\section{BIT - Fenwick Tree}

Soma $1..N$ e update em ponto em $O(log n)$.
\begin{multicols}{2}
	\begin{lstlisting}
#define MAXN 10000
int bit[MAXN];
void update(int x, int val){
	for(; x < MAXN; x+=x&-x)
		bit[x] += val;
}
int get(int x){
	int ans = 0;
	for(; x; x-=x&-x)
		ans += bit[x];
	return ans;
}
	\end{lstlisting}
\end{multicols}

\section{BIT - Fenwick Tree 2D}

Soma um subretângulo e update em ponto em $O(log^2n)$.
\begin{multicols}{2}
	\begin{lstlisting}
#define MAXN 1000
int bit[MAXN][MAXN];

void update(int x, int y, int val){
	for(; x < MAXN; x+=x&-x)
		for(int j = y; j < MAXN; j+=j&-j)
			bit[x][j] += val;
}

int get(int x, int y){
	int ans = 0;
	for(; x; x-=x&-x)
		for(int j = y; j; j-=j&-j)
			ans += bit[x][j];
	return ans;
}

int get(int x1, int y1, int x2, int y2){
	return get(x2, y2) - get(x1-1, y2) - get(x2, y1-1) + get(x1-1, y1-1);
}

	\end{lstlisting}
\end{multicols}

