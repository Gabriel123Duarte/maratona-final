\chapter{Estruturas de dados}

\section{Prefix Sum 1D}

Soma $a..b$ em $O(1)$.
\begin{multicols}{2}
\begin{lstlisting}
#define MAXN 1000
int arr[MAXN];
int prefix[MAXN];

void build(int n){
	prefix[0] = 0;
	for(int i = 1; i <= n; i++) // arr 1-indexado
		prefix[i] = prefix[i-1]+arr[i];
}

int get(int a, int b){
	return prefix[b] - prefix[a-1];
}

\end{lstlisting}
\end{multicols}


\section{BIT - Fenwick Tree}

Soma $1..N$ e update em ponto em $O(log n)$.
\begin{multicols}{2}
	\begin{lstlisting}
#define MAXN 10000
int bit[MAXN];
void update(int x, int val){
	for(; x < MAXN; x+=x&-x)
		bit[x] += val;
}
int get(int x){
	int ans = 0;
	for(; x; x-=x&-x)
		ans += bit[x];
	return ans;
}
	\end{lstlisting}
\end{multicols}

\section{BIT - Fenwick Tree 2D}

Soma um subretângulo e update em ponto em $O(log^2n)$.
\begin{multicols}{2}
	\begin{lstlisting}
#define MAXN 1000
int bit[MAXN][MAXN];

void update(int x, int y, int val){
	for(; x < MAXN; x+=x&-x)
		for(int j = y; j < MAXN; j+=j&-j)
			bit[x][j] += val;
}

int get(int x, int y){
	int ans = 0;
	for(; x; x-=x&-x)
		for(int j = y; j; j-=j&-j)
			ans += bit[x][j];
	return ans;
}

int get(int x1, int y1, int x2, int y2){
	return get(x2, y2) - get(x1-1, y2) - get(x2, y1-1) + get(x1-1, y1-1);
}

	\end{lstlisting}
\end{multicols}

\section{BIT - Fenwick Tree 2D - Range Update}

Update em range, consulta em ponto e em range.
\begin{multicols}{2}
	\begin{lstlisting}
#define MAXN 505

ll bit[4][MAXN + 50][MAXN + 50];

void update(int node, int x, int y, ll v){
	for(; x <= MAXN; x +=x&-x)
	for(int j = y; j <= MAXN; j+=j&-j)
	bit[node][x][j] += v;
}

ll query(int node, int x, int y){
	ll ans = 0;
	for(; x; x-=x&-x)
	for(int j = y; j; j-=j&-j)
	ans += bit[node][x][j];
	return ans;
}

void updateSubMatrix(int x1, int y1, int x2, int y2, ll val){
	update(0, x1, y1, val);
	update(0, x1, y2 + 1, -val);
	update(0, x2 + 1, y1, -val);	
	update(0, x2+1, y2+1, val);
	
	update(1, x1, y1, val*(1-y1));
	update(1, x1, y2+1, val*y2);
	update(1, x2+1, y1, val*(y1-1));
	update(1, x2+1, y2+1, -val*y2);
	
	update(2, x1, y1, val*(1-x1));
	update(2, x1, y2+1, (x1-1)*val);
	update(2, x2+1, y1, val*x2);
	update(2, x2+1, y2+1, -x2*val);
	
	update(3, x1, y1, (x1-1)*(y1-1)*val);
	update(3, x1, y2+1, -y2*(x1-1)*val);
	update(3, x2+1, y1, -x2*(y1-1)*val);
	update(3, x2+1, y2+1, x2*y2*val);
}

ll queryPoint(int x, int y){
	return query(0, x, y) * x * y + query(1, x, y) * x + query(2, x, y) * y + query(3, x, y);
}

ll querySubMatrix(int x1, int y1, int x2, int y2){
	return queryPoint(x2, y2) - queryPoint(x1 - 1, y2) - queryPoint(x2, y1 - 1) + queryPoint(x1 - 1, y1 - 1);
}

\end{lstlisting}
\end{multicols}

\section{Segment Tree 2D}

Quando a consulta é em uma distância de manhattan d, basta rotacionar o grid 45º.
Todo ponto (x, y) vira (x+y, x-y).
A consulta fica ((x+d, y+d), (x-d, y-d))

\begin{multicols}{2}
	\begin{lstlisting}

#define MAXN 1030

int tree[4*MAXN][4*MAXN];

void buildy(int idxx, int lx, int rx, int idxy, int ly, int ry){
	if(ly == ry){
		if(lx == rx)
			tree[idxx][idxy] = 0; // Valor inicial
		else
			tree[idxx][idxy] = tree[idxx*2][idxy] + tree[idxx*2+1][idxy];
		return;
	}
	buildy(idxx, lx, rx, idxy*2, ly, (ly+ry)/2);
	buildy(idxx, lx, rx, idxy*2+1, (ly+ry)/2+1, ry);
	tree[idxx][idxy] = tree[idxx][idxy*2] + tree[idxx][idxy*2+1];
}

void buildx(int idx, int lx, int rx, int ly, int ry){
	if(lx != rx){
		buildx(idx*2, lx, (lx+rx)/2, ly, ry);
		buildx(idx*2+1, (lx+rx)/2+1, rx, ly, ry);
	}
	buildy(idx, lx, rx, 1, ly, ry);
}

int gety(int idxx, int idxy, int ly, int ry, int y1, int y2){
	if(ly > y2 || ry < y1)
		return 0;
	if(ly >= y1 && ry <= y2)
		return tree[idxx][idxy];
	return gety(idxx, idxy*2, ly, (ly+ry)/2, y1, y2) + gety(idxx, idxy*2+1, (ly+ry)/2+1, ry, y1, y2);
}

int getx(int idxx, int lx, int rx, int idxy, int ly, int ry, int x1, int x2, int y1, int y2){
	if(lx > x2 || rx < x1)
		return 0;
	if(lx >= x1 && rx <= x2)
		return gety(idxx, idxy, ly, ry, y1, y2);
	return getx(idxx*2, lx, (lx+rx)/2, idxy, ly, ry, x1, x2, y1, y2) +
	getx(idxx*2+1, (lx+rx)/2+1, rx, idxy, ly, ry, x1, x2, y1, y2);
}

void updatey(int idxx, int lx, int rx, int idxy, int ly, int ry, int py, int val){
	if(ly > py || ry < py)
	return;
	if(ly == ry){
		if(lx == rx)
			tree[idxx][idxy] += val;
		else
			tree[idxx][idxy] = tree[idxx*2][idxy] + tree[idxx*2+1][idxy];
		return;
	}
	updatey(idxx, lx, rx, idxy*2, ly, (ly+ry)/2, py, val);
	updatey(idxx, lx, rx, idxy*2+1, (ly+ry)/2+1, ry, py, val);
	tree[idxx][idxy] = tree[idxx][idxy*2] + tree[idxx][idxy*2+1];
}

void updatex(int idxx, int lx, int rx, int idxy, int ly, int ry, int px, int py, int val){
	if(lx > px || rx < px)
	return;
	if(lx != rx){
		updatex(idxx*2, lx, (lx+rx)/2, idxy, ly, ry, px, py, val);
		updatex(idxx*2+1, (lx+rx)/2+1, rx, idxy, ly, ry, px, py, val);
	}
	updatey(idxx, lx, rx, idxy, ly, ry, py, val);
}

	\end{lstlisting}
\end{multicols}
\section{Kd-Tree}

Encontra os K pontos mais próximos de um dado ponto $O(k log(k) log(n))$.
\begin{multicols}{2}
	\begin{lstlisting}

#define MAXN 10100

double dist(int x, int y, int xx, int yy){
	return hypot(x - xx, y - yy);
}


// 2D point object
struct point {
	double x, y;
	point(double x = 0, double y = 0): x(x), y(y) {}	
};

// the "hyperplane split", use comparators for all dimensions
bool cmpx(const point& a, const point& b) {return a.x < b.x;}
bool cmpy(const point& a, const point& b) {return a.y < b.y;}

struct kdtree {
	point *tree;
	int n;
	// constructor
	kdtree(point p[], int n): tree(new point[n]), n(n) {
		copy(p, p + n, tree);
		build(0, n, false);
	}
	// destructor
	~kdtree() {delete[] tree;}
	// k-nearest neighbor query, O(k log(k) log(n)) on average
	vector<point> query(double x, double y, int k = 1) {
		perform_query(x, y, k, 0, n, false); // recurse
		vector<point> points;
		while (!pq.empty()) { // collect points
			points.push_back(*pq.top().second);
			pq.pop();
		}
		reverse(points.begin(), points.end());
		return points;
	}
	private:
	// build is O(n log n) using divide and conquer
	void build(int L, int R, bool dvx) {
		if (L >= R) return;
		int M = (L + R) / 2;
		// get median in O(n), split x-coordinate if dvx is true
		nth_element(tree+L, tree+M, tree+R, dvx?cmpx:cmpy);
		build(L, M, !dvx); build(M+1, R, !dvx);
	}
	
	// priority queue for KNN, keep the K nearest
	priority_queue<pair<double, point*> > pq;
	void perform_query(double x, double y, int k, int L, int R, bool dvx) {
		if (L >= R) return;
		int M = (L + R) / 2;
		double dx = x - tree[M].x;
		double dy = y - tree[M].y;
		double delta = dvx ? dx : dy;
		double dist = dx * dx + dy * dy;
		// if point is nearer to the kth farthest, put point in queue
		if (pq.size() < k || dist < pq.top().first) {
			pq.push(make_pair(dist, &tree[M]));
			if (pq.size() > k) pq.pop(); // keep k elements only
		}
		int nearL = L, nearR = M, farL = M + 1, farR = R;
		if (delta > 0) { // right is nearer
			swap(nearL, farL);
			swap(nearR, farR);
		}
		// query the nearer child
		perform_query(x, y, k, nearL, nearR, !dvx);
		
		if (pq.size() < k || delta * delta < pq.top().first)
		// query the farther child if there might be candidates
		perform_query(x, y, k, farL, farR, !dvx);
	}
};
\end{lstlisting}
\end{multicols}

\section{Treap}

Implicit cartesian tree $O(logn)$.
\begin{multicols}{2}
	\begin{lstlisting}
// Prior e size obrigatorios
// Carregar o que precisa
struct node{
	int prior, size, lazy;
	int val;
	
	node *l, *r;
	node() {}
	node(int n){
		prior = rand();
		size = 1;
		val = n;
		lazy = 0;
		l = r = NULL;
	}
};

int size(node *t){
	return t ? t->size : 0;
}

void updateSize(node *t){
	if(t)
	t->size = 1 + size(t->l) + size(t->r);
}

// Lazy para inverter intervalo
void lazy(node *t){
	if(!t || !t->lazy)
		return;
	
	t->lazy = t->lazy % 2;
	if(t->lazy){
		swap(t->r, t->l);
		if(t->l)
			t->l->lazy++;
		if(t->r)
			t->r->lazy++;
	}
	t->lazy = 0;
}

// Operator +
void operation(node *t){
	if(!t)
		return;
	lazy(t->l);
	lazy(t->r);
	
	
	if(t->l)
		t += juncao com filho esquerda
	if(t->r)
		t += juncao com filho direita
	
	t += informacao do no atual
}

void split(node *t, node *&l, node *&r, int pos, int add = 0){
	if(!t){
		l = r = NULL;
		return;
	}
	
	lazy(t);
	int cur_pos = add + size(t->l);
	if(cur_pos <= pos)
		split(t->r, t->r, r, pos, cur_pos + 1), l = t;
	else
		split(t->l, l, t->l, pos, add), r = t;
	updateSize(t);
	operation(t);
}

void merge(node *&t, node *l, node *r){
	lazy(l);
	lazy(r);
	if(!l || !r)
		t = l ? l : r;
	else if(l->prior > r->prior)
		merge(l->r, l->r, r), t = l;
	else
		merge(r->l, l, r->l), t = r;
	updateSize(t);
	operation(t);
}

// Inverte o range l..r
void inverter(node *t, int l, int r){
	node *L, *mid, *R;
	split(t, L, mid, l - 1);
	split(mid, t, R, r - l);
	t->lazy++;
	merge(mid, L, t);
	merge(t, mid, R);
}

// Criacao da Treap na main
node *Treap;
for(int i = 0; i < n; i++){
	if(!i)
		Treap = new node(v[i]);	
	else
		merge(Treap, Treap, new node(v[i]));
}
\end{lstlisting}
\end{multicols}


\section{Sparse table}

Suporta $min$, $max$, $gcd$, $lcm$, build em $O(nlogn)$ e query em $O(1)$.
\begin{multicols}{2}
	\begin{lstlisting}
#define MAXN 100100
#define LOG  17     // ~log2(MAXN)

int arr[MAXN], st[MAXN][LOG];

void build(int n){
	for(int i = 0; i < n; i++)
		st[i][0] = arr[i];
	
	for(int j = 1; (1 << j) <= n; j++)
		for(int i = 0; i + (1 << j) - 1 < n; i++)
			st[i][j] = min(st[i][j - 1], st[i + (1 << (j - 1))][j - 1]);
}

int query(int l, int r){
	// Pre processar os logs ou usar __builtin_ctz se o tempo estiver apertado
	int k = floor(log2((double)r - l + 1));
	return min(st[l][k], st[r - (1 << k) + 1][k]);
}
\end{lstlisting}
\end{multicols}

\section{Persistent Segment Tree}

Persistent aplicada para encontrar o menor elemento que não pode ser formado através da soma de elementos de um subarray. 
\begin{multicols}{2}
	\begin{lstlisting}

struct node{
	node *l, *r;
	ll sum;
	node(){
		l = r = 0;
		sum = 0;
	}	
};

ii v[MAXN];
node *roots[MAXN];
int n, q;

void update(node *last, node *cur, int l, int r, int pos, int val){
	if(l > pos || r < pos)
	return;
	if(l == r && r == pos){
		cur->sum = last->sum + val;
		return;
	}
	
	int mid = (l+r)/2;
	if(pos <= mid){
		cur->l = new node();
		cur->r = last->r;
		update(last->l, cur->l, l, (l+r)/2, pos, val);
	}
	else{
		cur->r = new node();
		cur->l = last->l;
		update(last->r, cur->r, (l+r)/2+1, r, pos, val);
	}
	cur->sum = cur->l->sum + cur->r->sum;
}

void build(node *cur, int l, int r){
	if(l == r)
	return;
	cur->l = new node();
	cur->r = new node();
	build(cur->l, l, (l+r)/2);
	build(cur->r, (l+r)/2+1, r);
}

ll get(node *cur, int l, int r, int x, int y){
	if(l > y || r < x)
	return 0;
	if(l >= x && r <= y)
	return cur->sum;
	return get(cur->l, l, (l+r)/2, x, y) + get(cur->r, (l+r)/2+1, r, x, y);
}

ll get(int l, int r){
	ll s = 0;
	while(1){
		ll cur = get(roots[upper_bound(v, v+n, ii(s+1, n+1)) - v], 1, n, l, r);
		if(cur == s)
		return s+1;
		s = cur;
	}
	return 0;
}

void solve(){
	scanf("%d %d", &n, &q);
	for(int i = 0; i < n; i++){
		scanf("%d", &v[i].fst);
		v[i].snd = i;
	}
	
	sort(v, v+n);
	
	roots[0] = new node();
	build(roots[0], 1, n);
	for(int i = 1; i <= n; i++){
		roots[i] = new node();
		update(roots[i-1], roots[i], 1, n, v[i-1].snd+1, v[i-1].fst);
	}
	
	int l, r;
	while(q--){
		scanf("%d %d", &l, &r);
		printf("%lld\n", get(l, r));
	}	
}
\end{lstlisting}
\end{multicols}