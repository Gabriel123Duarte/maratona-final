\chapter{Estruturas de dados}

\section{Prefix Sum 1D}

Soma $a..b$ em $O(1)$.
\begin{multicols}{2}
\begin{lstlisting}
#define MAXN 1000
int arr[MAXN];
int prefix[MAXN];

void build(int n){
	prefix[0] = 0;
	for(int i = 1; i <= n; i++) // arr 1-indexado
		prefix[i] = prefix[i-1]+arr[i];
}

int get(int a, int b){
	return prefix[b] - prefix[a-1];
}

\end{lstlisting}
\end{multicols}


\section{BIT - Fenwick Tree}

Soma $1..N$ e update em ponto em $O(log n)$.
\begin{multicols}{2}
	\begin{lstlisting}
#define MAXN 10000
int bit[MAXN];
void update(int x, int val){
	for(; x < MAXN; x+=x&-x)
		bit[x] += val;
}
int get(int x){
	int ans = 0;
	for(; x; x-=x&-x)
		ans += bit[x];
	return ans;
}
	\end{lstlisting}
\end{multicols}

\section{BIT - Fenwick Tree 2D}

Soma um subretângulo e update em ponto em $O(log^2n)$.
\begin{multicols}{2}
	\begin{lstlisting}
#define MAXN 1000
int bit[MAXN][MAXN];

void update(int x, int y, int val){
	for(; x < MAXN; x+=x&-x)
		for(int j = y; j < MAXN; j+=j&-j)
			bit[x][j] += val;
}

int get(int x, int y){
	int ans = 0;
	for(; x; x-=x&-x)
		for(int j = y; j; j-=j&-j)
			ans += bit[x][j];
	return ans;
}

int get(int x1, int y1, int x2, int y2){
	return get(x2, y2) - get(x1-1, y2) - get(x2, y1-1) + get(x1-1, y1-1);
}

	\end{lstlisting}
\end{multicols}

\section{BIT - Fenwick Tree 2D - Range Update}

Update em range, consulta em ponto e em range.
\begin{multicols}{2}
	\begin{lstlisting}
#define MAXN 505

ll bit[4][MAXN + 50][MAXN + 50];

void update(int node, int x, int y, ll v){
	for(; x <= MAXN; x +=x&-x)
	for(int j = y; j <= MAXN; j+=j&-j)
	bit[node][x][j] += v;
}

ll query(int node, int x, int y){
	ll ans = 0;
	for(; x; x-=x&-x)
	for(int j = y; j; j-=j&-j)
	ans += bit[node][x][j];
	return ans;
}

void updateSubMatrix(int x1, int y1, int x2, int y2, ll val){
	update(0, x1, y1, val);
	update(0, x1, y2 + 1, -val);
	update(0, x2 + 1, y1, -val);	
	update(0, x2+1, y2+1, val);
	
	update(1, x1, y1, val*(1-y1));
	update(1, x1, y2+1, val*y2);
	update(1, x2+1, y1, val*(y1-1));
	update(1, x2+1, y2+1, -val*y2);
	
	update(2, x1, y1, val*(1-x1));
	update(2, x1, y2+1, (x1-1)*val);
	update(2, x2+1, y1, val*x2);
	update(2, x2+1, y2+1, -x2*val);
	
	update(3, x1, y1, (x1-1)*(y1-1)*val);
	update(3, x1, y2+1, -y2*(x1-1)*val);
	update(3, x2+1, y1, -x2*(y1-1)*val);
	update(3, x2+1, y2+1, x2*y2*val);
}

ll queryPoint(int x, int y){
	return query(0, x, y) * x * y + query(1, x, y) * x + query(2, x, y) * y + query(3, x, y);
}

ll querySubMatrix(int x1, int y1, int x2, int y2){
	return queryPoint(x2, y2) - queryPoint(x1 - 1, y2) - queryPoint(x2, y1 - 1) + queryPoint(x1 - 1, y1 - 1);
}

\end{lstlisting}
\end{multicols}

\section{Segment Tree 2D}

Quando a consulta é em uma distância de manhattan d, basta rotacionar o grid 45º.
Todo ponto (x, y) vira (x+y, x-y).
A consulta fica ((x+d, y+d), (x-d, y-d))

\begin{multicols}{2}
	\begin{lstlisting}

#define MAXN 1030

int tree[4*MAXN][4*MAXN];

void buildy(int idxx, int lx, int rx, int idxy, int ly, int ry){
	if(ly == ry){
		if(lx == rx)
			tree[idxx][idxy] = 0; // Valor inicial
		else
			tree[idxx][idxy] = tree[idxx*2][idxy] + tree[idxx*2+1][idxy];
		return;
	}
	buildy(idxx, lx, rx, idxy*2, ly, (ly+ry)/2);
	buildy(idxx, lx, rx, idxy*2+1, (ly+ry)/2+1, ry);
	tree[idxx][idxy] = tree[idxx][idxy*2] + tree[idxx][idxy*2+1];
}

void buildx(int idx, int lx, int rx, int ly, int ry){
	if(lx != rx){
		buildx(idx*2, lx, (lx+rx)/2, ly, ry);
		buildx(idx*2+1, (lx+rx)/2+1, rx, ly, ry);
	}
	buildy(idx, lx, rx, 1, ly, ry);
}

int gety(int idxx, int idxy, int ly, int ry, int y1, int y2){
	if(ly > y2 || ry < y1)
		return 0;
	if(ly >= y1 && ry <= y2)
		return tree[idxx][idxy];
	return gety(idxx, idxy*2, ly, (ly+ry)/2, y1, y2) + gety(idxx, idxy*2+1, (ly+ry)/2+1, ry, y1, y2);
}

int getx(int idxx, int lx, int rx, int idxy, int ly, int ry, int x1, int x2, int y1, int y2){
	if(lx > x2 || rx < x1)
		return 0;
	if(lx >= x1 && rx <= x2)
		return gety(idxx, idxy, ly, ry, y1, y2);
	return getx(idxx*2, lx, (lx+rx)/2, idxy, ly, ry, x1, x2, y1, y2) +
	getx(idxx*2+1, (lx+rx)/2+1, rx, idxy, ly, ry, x1, x2, y1, y2);
}

void updatey(int idxx, int lx, int rx, int idxy, int ly, int ry, int py, int val){
	if(ly > py || ry < py)
	return;
	if(ly == ry){
		if(lx == rx)
			tree[idxx][idxy] += val;
		else
			tree[idxx][idxy] = tree[idxx*2][idxy] + tree[idxx*2+1][idxy];
		return;
	}
	updatey(idxx, lx, rx, idxy*2, ly, (ly+ry)/2, py, val);
	updatey(idxx, lx, rx, idxy*2+1, (ly+ry)/2+1, ry, py, val);
	tree[idxx][idxy] = tree[idxx][idxy*2] + tree[idxx][idxy*2+1];
}

void updatex(int idxx, int lx, int rx, int idxy, int ly, int ry, int px, int py, int val){
	if(lx > px || rx < px)
	return;
	if(lx != rx){
		updatex(idxx*2, lx, (lx+rx)/2, idxy, ly, ry, px, py, val);
		updatex(idxx*2+1, (lx+rx)/2+1, rx, idxy, ly, ry, px, py, val);
	}
	updatey(idxx, lx, rx, idxy, ly, ry, py, val);
}

	\end{lstlisting}
\end{multicols}

\section{Convex hull trick 1}

Quando o X está ordenado. \\
Inserir retas do tipo Y = A*X + B. \\
Para máximo adicionar o A e B negativos e quando consultar coloca negativo o valor.

\begin{multicols}{2}
	\begin{lstlisting}
struct hull{
	ll A[MAXN];
	ll B[MAXN];
	int len, ptr;
	hull(){
		len = ptr = 0;
	}
	
	void addLine(ll a, ll b){
		while(len >= 2 && (B[len-2] - B[len-1]) * (a-A[len-1]) >= (B[len-1]-b) * (A[len-1]-A[len-2]))
		len--;
		A[len] = a;
		B[len] = b;
		len++;
	}
	
	ll get(ll x){
		ptr = min(ptr, len-1);
		while(ptr+1 < len && A[ptr+1]*x+B[ptr+1] <= A[ptr]*x + B[ptr])
		ptr++;
		return A[ptr]*x + B[ptr];
	}
};

\end{lstlisting}
\end{multicols}

\section{Convex hull trick 2}

Quando o X não está ordenado. 

\begin{multicols}{2}
	\begin{lstlisting}

class ConvexHullTrick {
	struct CHTPoint {
		double x, y, lim;
		
	};
	vector<CHTPoint> pilha;
	inline double get_intersection(CHTPoint a, CHTPoint b) {
		double denom = ( double(b.x) - a.x);
		double num = ( double(b.y) - a.y);
		
		return -num / denom;
	}
	
	bool ccw(CHTPoint p0, CHTPoint p1, CHTPoint p2) {
		return ((double)(p1.y-p0.y)*(double)(p2.x-p0.x) > (double)(p2.y-p0.y)*(double)(p1.x-p0.x));
	}
	
	public:
	
	void add_line(double a, double b) {
		CHTPoint novo = {a, b, 0};
		int tam = pilha.size();
		while (tam >= 2 && !ccw(pilha[tam-2], pilha[tam-1], novo)) {
			pilha.pop_back();
			tam--;
		}
		while (tam >= 1 && fabs(pilha[tam-1].x - a) < 1e-8) {
			pilha.pop_back();
			tam--;
		}
		
		pilha.push_back(novo);
		if (tam >= 1) pilha[tam-1].lim = get_intersection(pilha[tam-1], pilha[tam]);
	}
	
	double get_maximum(double x) {
		int st = 0, ed = pilha.size() - 1;
		while (st < ed) {
			int mid = (st+ed)/2;
			if (pilha[mid].lim < x) st = mid+1;
			else ed = mid;
		}
		return pilha[st].x * x + pilha[st].y;
	}
	
};


\end{lstlisting}
\end{multicols}

\section{Convex hull trick 3 - Fully Dynamic}

Sem condições especiais para o A e B

\begin{multicols}{2}
	\begin{lstlisting}


const ll is_query = -(1LL<<62);
struct Line {
	ll m, b;
	mutable function<const Line*()> succ;
	bool operator<(const Line& rhs) const {
		if (rhs.b != is_query) return m < rhs.m;
		const Line* s = succ();
		if (!s) return 0;
		ll x = rhs.m;
		return b - s->b < (s->m - m) * x;
	}
};
struct HullDynamic : public multiset<Line> { // will maintain upper hull for maximum
	bool bad(iterator y) {
		auto z = next(y);
		if (y == begin()) {
			if (z == end()) return 0;
			return y->m == z->m && y->b <= z->b;
		}
		auto x = prev(y);
		if (z == end()) return y->m == x->m && y->b <= x->b;
		return (x->b - y->b)*(z->m - y->m) >= (y->b - z->b)*(y->m - x->m);
	}
	void insert_line(ll m, ll b) {
		auto y = insert({ m, b });
		y->succ = [=] { return next(y) == end() ? 0 : &*next(y); };
		if (bad(y)) { erase(y); return; }
		while (next(y) != end() && bad(next(y))) erase(next(y));
		while (y != begin() && bad(prev(y))) erase(prev(y));
	}
	ll eval(ll x) {
		auto l = *lower_bound((Line) { x, is_query });
		return l.m * x + l.b;
	}
};

\end{lstlisting}
\end{multicols}

\section{Kd-Tree}

Encontra os K pontos mais próximos de um dado ponto $O(k log(k) log(n))$.
\begin{multicols}{2}
	\begin{lstlisting}

#define MAXN 10100

double dist(int x, int y, int xx, int yy){
	return hypot(x - xx, y - yy);
}


// 2D point object
struct point {
	double x, y;
	point(double x = 0, double y = 0): x(x), y(y) {}	
};

// the "hyperplane split", use comparators for all dimensions
bool cmpx(const point& a, const point& b) {return a.x < b.x;}
bool cmpy(const point& a, const point& b) {return a.y < b.y;}

struct kdtree {
	point *tree;
	int n;
	// constructor
	kdtree(point p[], int n): tree(new point[n]), n(n) {
		copy(p, p + n, tree);
		build(0, n, false);
	}
	// destructor
	~kdtree() {delete[] tree;}
	// k-nearest neighbor query, O(k log(k) log(n)) on average
	vector<point> query(double x, double y, int k = 1) {
		perform_query(x, y, k, 0, n, false); // recurse
		vector<point> points;
		while (!pq.empty()) { // collect points
			points.push_back(*pq.top().second);
			pq.pop();
		}
		reverse(points.begin(), points.end());
		return points;
	}
	private:
	// build is O(n log n) using divide and conquer
	void build(int L, int R, bool dvx) {
		if (L >= R) return;
		int M = (L + R) / 2;
		// get median in O(n), split x-coordinate if dvx is true
		nth_element(tree+L, tree+M, tree+R, dvx?cmpx:cmpy);
		build(L, M, !dvx); build(M+1, R, !dvx);
	}
	
	// priority queue for KNN, keep the K nearest
	priority_queue<pair<double, point*> > pq;
	void perform_query(double x, double y, int k, int L, int R, bool dvx) {
		if (L >= R) return;
		int M = (L + R) / 2;
		double dx = x - tree[M].x;
		double dy = y - tree[M].y;
		double delta = dvx ? dx : dy;
		double dist = dx * dx + dy * dy;
		// if point is nearer to the kth farthest, put point in queue
		if (pq.size() < k || dist < pq.top().first) {
			pq.push(make_pair(dist, &tree[M]));
			if (pq.size() > k) pq.pop(); // keep k elements only
		}
		int nearL = L, nearR = M, farL = M + 1, farR = R;
		if (delta > 0) { // right is nearer
			swap(nearL, farL);
			swap(nearR, farR);
		}
		// query the nearer child
		perform_query(x, y, k, nearL, nearR, !dvx);
		
		if (pq.size() < k || delta * delta < pq.top().first)
		// query the farther child if there might be candidates
		perform_query(x, y, k, farL, farR, !dvx);
	}
};


\end{lstlisting}
\end{multicols}

